% Options for packages loaded elsewhere
\PassOptionsToPackage{unicode}{hyperref}
\PassOptionsToPackage{hyphens}{url}
\PassOptionsToPackage{dvipsnames,svgnames,x11names}{xcolor}
%
\documentclass[
  letterpaper,
  DIV=11,
  numbers=noendperiod]{scrreprt}

\usepackage{amsmath,amssymb}
\usepackage{lmodern}
\usepackage{iftex}
\ifPDFTeX
  \usepackage[T1]{fontenc}
  \usepackage[utf8]{inputenc}
  \usepackage{textcomp} % provide euro and other symbols
\else % if luatex or xetex
  \usepackage{unicode-math}
  \defaultfontfeatures{Scale=MatchLowercase}
  \defaultfontfeatures[\rmfamily]{Ligatures=TeX,Scale=1}
\fi
% Use upquote if available, for straight quotes in verbatim environments
\IfFileExists{upquote.sty}{\usepackage{upquote}}{}
\IfFileExists{microtype.sty}{% use microtype if available
  \usepackage[]{microtype}
  \UseMicrotypeSet[protrusion]{basicmath} % disable protrusion for tt fonts
}{}
\makeatletter
\@ifundefined{KOMAClassName}{% if non-KOMA class
  \IfFileExists{parskip.sty}{%
    \usepackage{parskip}
  }{% else
    \setlength{\parindent}{0pt}
    \setlength{\parskip}{6pt plus 2pt minus 1pt}}
}{% if KOMA class
  \KOMAoptions{parskip=half}}
\makeatother
\usepackage{xcolor}
\setlength{\emergencystretch}{3em} % prevent overfull lines
\setcounter{secnumdepth}{5}
% Make \paragraph and \subparagraph free-standing
\ifx\paragraph\undefined\else
  \let\oldparagraph\paragraph
  \renewcommand{\paragraph}[1]{\oldparagraph{#1}\mbox{}}
\fi
\ifx\subparagraph\undefined\else
  \let\oldsubparagraph\subparagraph
  \renewcommand{\subparagraph}[1]{\oldsubparagraph{#1}\mbox{}}
\fi


\providecommand{\tightlist}{%
  \setlength{\itemsep}{0pt}\setlength{\parskip}{0pt}}\usepackage{longtable,booktabs,array}
\usepackage{calc} % for calculating minipage widths
% Correct order of tables after \paragraph or \subparagraph
\usepackage{etoolbox}
\makeatletter
\patchcmd\longtable{\par}{\if@noskipsec\mbox{}\fi\par}{}{}
\makeatother
% Allow footnotes in longtable head/foot
\IfFileExists{footnotehyper.sty}{\usepackage{footnotehyper}}{\usepackage{footnote}}
\makesavenoteenv{longtable}
\usepackage{graphicx}
\makeatletter
\def\maxwidth{\ifdim\Gin@nat@width>\linewidth\linewidth\else\Gin@nat@width\fi}
\def\maxheight{\ifdim\Gin@nat@height>\textheight\textheight\else\Gin@nat@height\fi}
\makeatother
% Scale images if necessary, so that they will not overflow the page
% margins by default, and it is still possible to overwrite the defaults
% using explicit options in \includegraphics[width, height, ...]{}
\setkeys{Gin}{width=\maxwidth,height=\maxheight,keepaspectratio}
% Set default figure placement to htbp
\makeatletter
\def\fps@figure{htbp}
\makeatother
\newlength{\cslhangindent}
\setlength{\cslhangindent}{1.5em}
\newlength{\csllabelwidth}
\setlength{\csllabelwidth}{3em}
\newlength{\cslentryspacingunit} % times entry-spacing
\setlength{\cslentryspacingunit}{\parskip}
\newenvironment{CSLReferences}[2] % #1 hanging-ident, #2 entry spacing
 {% don't indent paragraphs
  \setlength{\parindent}{0pt}
  % turn on hanging indent if param 1 is 1
  \ifodd #1
  \let\oldpar\par
  \def\par{\hangindent=\cslhangindent\oldpar}
  \fi
  % set entry spacing
  \setlength{\parskip}{#2\cslentryspacingunit}
 }%
 {}
\usepackage{calc}
\newcommand{\CSLBlock}[1]{#1\hfill\break}
\newcommand{\CSLLeftMargin}[1]{\parbox[t]{\csllabelwidth}{#1}}
\newcommand{\CSLRightInline}[1]{\parbox[t]{\linewidth - \csllabelwidth}{#1}\break}
\newcommand{\CSLIndent}[1]{\hspace{\cslhangindent}#1}

\KOMAoption{captions}{tableheading}
\makeatletter
\makeatother
\makeatletter
\@ifpackageloaded{bookmark}{}{\usepackage{bookmark}}
\makeatother
\makeatletter
\@ifpackageloaded{caption}{}{\usepackage{caption}}
\AtBeginDocument{%
\ifdefined\contentsname
  \renewcommand*\contentsname{Table of contents}
\else
  \newcommand\contentsname{Table of contents}
\fi
\ifdefined\listfigurename
  \renewcommand*\listfigurename{List of Figures}
\else
  \newcommand\listfigurename{List of Figures}
\fi
\ifdefined\listtablename
  \renewcommand*\listtablename{List of Tables}
\else
  \newcommand\listtablename{List of Tables}
\fi
\ifdefined\figurename
  \renewcommand*\figurename{Figure}
\else
  \newcommand\figurename{Figure}
\fi
\ifdefined\tablename
  \renewcommand*\tablename{Table}
\else
  \newcommand\tablename{Table}
\fi
}
\@ifpackageloaded{float}{}{\usepackage{float}}
\floatstyle{ruled}
\@ifundefined{c@chapter}{\newfloat{codelisting}{h}{lop}}{\newfloat{codelisting}{h}{lop}[chapter]}
\floatname{codelisting}{Listing}
\newcommand*\listoflistings{\listof{codelisting}{List of Listings}}
\makeatother
\makeatletter
\@ifpackageloaded{caption}{}{\usepackage{caption}}
\@ifpackageloaded{subcaption}{}{\usepackage{subcaption}}
\makeatother
\makeatletter
\@ifpackageloaded{tcolorbox}{}{\usepackage[many]{tcolorbox}}
\makeatother
\makeatletter
\@ifundefined{shadecolor}{\definecolor{shadecolor}{rgb}{.97, .97, .97}}
\makeatother
\makeatletter
\makeatother
\ifLuaTeX
  \usepackage{selnolig}  % disable illegal ligatures
\fi
\IfFileExists{bookmark.sty}{\usepackage{bookmark}}{\usepackage{hyperref}}
\IfFileExists{xurl.sty}{\usepackage{xurl}}{} % add URL line breaks if available
\urlstyle{same} % disable monospaced font for URLs
\hypersetup{
  pdftitle={Kvantitatív Alapok},
  pdfauthor={Kovács Márton},
  colorlinks=true,
  linkcolor={blue},
  filecolor={Maroon},
  citecolor={Blue},
  urlcolor={Blue},
  pdfcreator={LaTeX via pandoc}}

\title{Kvantitatív Alapok}
\author{Kovács Márton}
\date{8/27/23}

\begin{document}
\maketitle
\ifdefined\Shaded\renewenvironment{Shaded}{\begin{tcolorbox}[borderline west={3pt}{0pt}{shadecolor}, boxrule=0pt, breakable, frame hidden, sharp corners, interior hidden, enhanced]}{\end{tcolorbox}}\fi

\renewcommand*\contentsname{Table of contents}
{
\hypersetup{linkcolor=}
\setcounter{tocdepth}{2}
\tableofcontents
}
\bookmarksetup{startatroot}

\hypertarget{elux151szuxf3}{%
\chapter*{Előszó}\label{elux151szuxf3}}
\addcontentsline{toc}{chapter}{Előszó}

\markboth{Előszó}{Előszó}

Ez a könyv a Neumann János Egyetem és az MNB Intézet által indított
Nemzetközi Gazdaság és Gazdálkodás mesterszakának \emph{Kvantitatív
alapok} nevű tantárgyához készült. Célom a hallgatók felkészülését
segíteni azzal, hogy egy helyen, átlátható formában közlöm a tantárgyhoz
készült jegyzeteket, példákat, gyakorlatokat és ábrákat. A Kvantitatív
alapok tantárgy felépítésénél akárcsak ezen könyv megírásánál nagyban
merítettem Russ Poldrack Statistical Thinking for the 21st Century című
munkájából, Nagy Tamás Adatelemzés és statisztikai következtetés nevű
órai jegyzeteiből és Andy Field statisztika tankönyveiből. Ha az olvasó
ezen művekből ismerős gondolatokat, példákat vagy magyarázatokat lát,
akkor azok nagy valószínűséggel onnan is lettek átvéve. Én magam is
nagyrészt tőlük tanultam azt, amit a kvantitatív elemzésről tudok. Ez a
három ember nálam sokkal többet gondolkozott azon, hogyan kell hatékony
módon statisztikát tanítani egyetemista hallgatóknak, így nem láttam
értelmét, hogy a tantárgy felépítését a nulláról kezdjem. Ezúton is
szeretném megköszönni a segítségük!

\bookmarksetup{startatroot}

\hypertarget{kuxf6szuxf6netnyuxedlvuxe1nuxedtuxe1s}{%
\chapter*{Köszönetnyílvánítás}\label{kuxf6szuxf6netnyuxedlvuxe1nuxedtuxe1s}}
\addcontentsline{toc}{chapter}{Köszönetnyílvánítás}

\markboth{Köszönetnyílvánítás}{Köszönetnyílvánítás}

Szeretnék továbbá köszönetet mondani Ignits Györgyinek, Kapitány
Balázsnak, Szászi Barnabásnak és Szécsi Péternek, akik példa adatok
megosztásán keresztül segítették a könyv létrejöttét.

A KÖNYVÖN MOST IS AKTÍVAN DOLGOZOM, EZ NEM A VÉGLEGES VÁLTOZAT!

\bookmarksetup{startatroot}

\hypertarget{bevezetuxe9s}{%
\chapter{Bevezetés}\label{bevezetuxe9s}}

A kvantitatív elemzés nem az adatoknál kezdődik. Az eddigi oktatói
tapasztalataim alapján amikor egy hallgató elakad egy önálló elemzési
feladat során az nem azért van mert helytelenül magolta be az éppen
alkalmazott statisztikai teszt előfeltételeit, vagy mert nem tudja a
standard hiba képletét. Ezekre a kérdésekre egy gyors Google kereséssel
könnyen meg lehet találni a választ.

\begin{itemize}
\item
  Hanem mert nem látja át az empirikus tudományos módszer egészét, annak
  lépéseit és összefüggéseit

  \begin{itemize}
  \item
    Nincs jól megfogalmazott hipotézis
  \item
    Nem feltett kutatási kérdés megválaszolásához szükséges
    mérőeszközöket használtak
  \item
    Nem látják át a rosszul strukturált adattáblát
  \item
    Nem értik a frekventista hipotézis tesztelés működését
  \item
    És csak ezután jön, hogy nem értik milyen statisztikai eljárást kell
    alkalmazni és azt hogyan
  \item
    Ezzel a lépéssel kapcsolatban én is gyakran teszek fel kérdéseket
    kollégáknak és az internetnek (és bíztatok mindenkit, hogy tegyen
    ugyanígy)
  \end{itemize}
\end{itemize}

\bookmarksetup{startatroot}

\hypertarget{tudomuxe1nyos-muxf3dszer}{%
\chapter{Tudományos módszer}\label{tudomuxe1nyos-muxf3dszer}}

\hypertarget{a-tudomuxe1ny-cuxe9lja}{%
\section{A tudomány célja}\label{a-tudomuxe1ny-cuxe9lja}}

\begin{itemize}
\item
  A tudomány célja, hogy általános magyarázotokkal szolgáljon a minket
  körülvevő világról
\item
  A tudományos magyarázatok leegyszerüsített formában \textbf{írják le}
  a valóságot
\item
  A tudományos magyarázatok segítségével így képesek vagyunk

  \begin{itemize}
  \item
    jól informált \textbf{döntéseket} hozni
  \item
    \textbf{predikciókat} alkotni új helyzetekre a már meglévő tudásunk
    alapján
  \end{itemize}
\item
  Később majd látni fogjuk, hogy a kutatási folyamat sok pontján
  félrecsúszhat az, hogy a kutatásunk alapján megbízható döntéseket
  hozzunk vagy predikciókat alkossunk
\item
  A tudományos megközelítés azonban nem az egyetlen mód, amelyen
  keresztül a valóságot magyarázzuk
\item
  Miben más, mint egyéb megismerési módok?

  \begin{itemize}
  \item
    Pl: vallás -\textgreater{} autoritás alapú
  \item
    Kritikai megközelítés
  \item
    Tesztelhető (falszifikálható hipotézisek
  \end{itemize}
\end{itemize}

\hypertarget{az-empirikus-tudomuxe1nyos-folyamat-luxe9puxe9sei}{%
\section{Az empirikus tudományos folyamat
lépései}\label{az-empirikus-tudomuxe1nyos-folyamat-luxe9puxe9sei}}

\begin{enumerate}
\def\labelenumi{\arabic{enumi}.}
\item
  Általános kérdésfelvetés

  \begin{itemize}
  \tightlist
  \item
    Ez talán a legegyszerűbb lépés, az ember kíváncsi természetű,
    szeretné érteni az őt körülvevő valóságot
  \end{itemize}
\item
  Elmélet alkotás

  \begin{itemize}
  \item
    Ahhoz, hogy egy kérdést meg tudjunk válaszolni empirikus módon egy
    elméletet kell alkotnunk a kérdés mögött álló lehetséges
    hatásmechanizmusról
  \item
    Az elmélet falszifikálható kell legyen

    \begin{itemize}
    \tightlist
    \item
      Tudunk olyan megfigyeléseket tenni amelyek fényében az elméletet
      elvetjük
    \end{itemize}
  \item
    A jó elméletek

    \begin{itemize}
    \item
      Elérés (reach): minél több jelenséget magyaráz meg egy elmélet
      annál jobb
    \item
      Parszimónia: két versenyző elmélet közül az a jobb, ami kevesebb
      elemből áll

      \begin{itemize}
      \tightlist
      \item
        Lásd Occam borotvája
      \end{itemize}
    \item
      Konzervatívizmus: Mennyire illeszkedik már meglévő tesztelt és
      elfogadott elméletekhez
    \end{itemize}
  \end{itemize}
\item
  Hipotézis alkotás

  \begin{itemize}
  \item
    A hipotézis az elméletünknek egy konkrétan tesztelhető
    operacionalizált része
  \item
    Egy elmélet alapján többféle hipotézist fel lehet állítani
  \item
    A hipotézis alapján predikciókat alkotunk, amelyeket tesztelünk
  \item
    Egy hipotézis bizonyításával nem bizonyítjuk az elméletet is!
  \item
    De tudjuk a teszt eredményei által tovább finomítani az elméletet
  \item
    Vagy elvetjük azt
  \item
    Fontos, hogy a hipotézis valóban illeszkedjen az adott elmélethez
  \end{itemize}
\item
  Kutatási elrendezés

  \begin{itemize}
  \item
    A hipotézis tesztelésére annak feltételezései alapján egy kutatást
    tervezünk
  \item
    A kutatás megtervezésénél számos különböző módszertan közül tudunk
    választani
  \item
    Mindegyiknek vannak előnyei és hátránya, a hipotézis és az elérhető
    erőforrások fényében választunk közülük

    \begin{itemize}
    \item
      Kérdőíves
    \item
      Megfigyeléses
    \item
      Kísérleti
    \end{itemize}
  \item
    Van pár általános szempont, amelyeket módszertantól függetlenül
    figyelembe kell vennünk és amelyek befolyásolják az eredményekből
    levontható következtetéseket

    \begin{itemize}
    \item
      Minta felépítése és mintaméret
    \item
      Mérőeszköz és annak pontossága, megbízhatósága, és granularitása
    \end{itemize}
  \item
    Ahhoz hogy oksági viszonyt feltételezzünk két jelenség közt nem
    feletkezhetünk meg az adatgyűjtés módjáról
  \item
    Megfigyeléses vizsgálatoknál nem tudjuk biztosan kijelenteni hogy
    két változó között oksági viszony van

    \begin{itemize}
    \item
      Lehetséges hogy egyébb faktorok befolyásolják a kapott
      együttjárást
    \item
      \url{https://www.tylervigen.com/spurious-correlations}
    \end{itemize}
  \item
    Ezért a kutatók általában az oksági viszony meghatározásához
    kísérleti kutatási elrendezést használnak (pl: randomized control
    trial)

    \begin{itemize}
    \tightlist
    \item
      A lehetséges, de nem viszgált befolyásoló faktorokat random
      mintavételezés útján zárják ki az adatgyűjtés során
    \end{itemize}
  \item
    Vannak statisztikai módszerek amivel utólag is kontrollálni tudjuk
    ezeket a változókat, de sokkal nehezebb ezt utólag megtenni
  \item
    Ez is kiemeli a jól megtervezett kutatások és részletesen
    kidolgozott elméletek fontosságát!
  \item
    Mintavételezés

    \begin{itemize}
    \item
      Az esetek kis részében tudjuk az egész populációt vizsgálni
    \item
      Minták tesztelésén keresztül vonunk le általános következtetéseket
    \item
      De mekkora mintára van szükség ahhoz, hogy a populációra tudjunk
      következtetni?
    \end{itemize}
  \end{itemize}
\item
  Adatok feldolgozása

  \begin{itemize}
  \item
    Általában nem szokták külön lépésként kezelni, de az óra témaja és
    az eredményekre gyakorolt lehetséges hatása miatt fontos megemlíteni
  \item
    A kapott adatokat befolyásolja az általunk választott kutatási
    elrendezés és az adatgyűjtés során fellépő váratlan torzító tényezők
  \item
    Nagyon ritka esetben készek az adatgyűjtés után az adatok az
    elemzésre
  \item
    Általában először adat rendezést és adattisztítást kell elvégeznünk
  \end{itemize}
\item
  Statisztikai elemzés

  \begin{itemize}
  \item
    Az elemzés során használhatunk

    \begin{itemize}
    \item
      Leíró statisztikai eljárásokat
    \item
      Következtetéses statisztikai eljárásokat
    \end{itemize}
  \item
    Rengeteg statisztikai eljárás van, a statisztikusok most is
    dolgoznak újak kifejlesztésén és validációján
  \item
    Az, hogy milyen eljárást választunk függ a hipotézisünktől, a
    kutatási elrendezésünktől, és az adatgyűjtés során kapott adatoktól
  \item
    Többféle lehetséges valid elemzési út is létezik

    \begin{itemize}
    \tightlist
    \item
      Lást: multi-analyst study
    \end{itemize}
  \end{itemize}
\item
  Eredmények értelmezése

  \begin{itemize}
  \item
    A statisztikai elemzés eredményeinek értelmezésére a tudományos
    folyamat összes eddigi lépése kihat
  \item
    Kutatóként szeretnénk egyértelmű válaszokat kapni az általunk
    feltett kérdésekre

    \begin{itemize}
    \item
      Sajnos nagyon kevés esetben ez a helyzet
    \item
      Ez részben a tudományos folyamat összetettségének az eredménye
    \item
      Részben ennek a vágynak is szerepe van a p-értékek abuzálásában
    \item
      Egy kutatásból nagyon ritkán kapunk egyértelmű válaszokat

      \begin{itemize}
      \tightlist
      \item
        Lásd: metaanalízis, szisztematikus összefoglaló (systematic
        review)
      \end{itemize}
    \end{itemize}
  \item
    Pár példa helyzet arra, amikor a tudományos módszer összetettsége
    árnyalja a statisztikai elemzés eredményeinek értelmezését:

    \begin{itemize}
    \item
      A választott mérőeszközök valójában nem a vizsgált konstruktumot
      mérik
    \item
      A minta milyensége vagy mérete nem indokolja az eredmények
      általánosítását
    \item
      Az adattisztítás során olyan megfigyelések is bent maradtak,
      amelyek szisztematikusan torzítják az eredményeket
    \end{itemize}
  \item
    Az értelmezést korlázotó tényezőkről általában a tudományos
    publikáció limitációk szakaszában számolunk be
  \end{itemize}
\end{enumerate}

\hypertarget{a-statisztika-szerepe-a-tudomuxe1nyban}{%
\section{A statisztika szerepe a
tudományban}\label{a-statisztika-szerepe-a-tudomuxe1nyban}}

\begin{itemize}
\item
  A komplex valóság egyszerüsített leírása, ami egyúttal azt is elmondja
  mennyire lehetünk bizonytalanok ebben a tudásban
\item
  Miért van szükség a statisztikára?

  \begin{itemize}
  \item
    Heurisztikák veszélye

    \begin{itemize}
    \item
      Példa: USA bűnelkövetések gyakorisága

      \begin{itemize}
      \item
        Sokkal gyakoribbnak gondoljuk a bűnelkövetések számát, mint
        amilyen gyakoriak a valóságban
      \item
        Elérhetőségi torzítás (kognitív torzítás egyik fajtája)
      \item
        Média szerepe?
      \end{itemize}
    \end{itemize}
  \item
    Statisztika segít abban hogy tanuljunk az adatokból

    \begin{itemize}
    \tightlist
    \item
      Hogyan frissítsük előzetes tudásunk az új adatok fényében
    \end{itemize}
  \end{itemize}
\item
  A statisztika az adatösszesítésről szól

  \begin{itemize}
  \tightlist
  \item
    Hogyan tudunk úgy leegyszerüsíteni befogadhatatlan mennyiségű
    megfigyelést, hogy azok segítségével hipotézisünk tesztelni tudjuk,
    de fontos információ ne vesszen el
  \end{itemize}
\item
  A statisztika a bizonytalanság kezeléséről szól

  \begin{itemize}
  \item
    A világ összetett
  \item
    A legtöbb összetett jelenséget nehéz determinisztikus módon
    megmagyarázni
  \item
    Bár tudunk ok-okozati kapcsolatot felállítani bizonyos összefüggések
    között általában ezek nem determinisztikus, hanem probabilisztikus
    predikciókhoz vezetnek

    \begin{itemize}
    \tightlist
    \item
      Lásd: dohányzás és tüdőrák

      \begin{itemize}
      \tightlist
      \item
        Tudjuk, hogy a dohányzás növeli a tüdőrák esélyét, azonban nem
        lehet biztosra mondani, hogyha valaki dohányzik, akkor
        mindenképp tüdőrákos is lesz
      \end{itemize}
    \end{itemize}
  \item
    A statisztika nem tud bizonyítani (úgy ahogy a matematika tud)

    \begin{itemize}
    \tightlist
    \item
      Hanem evidenciát tud felmutatni egy hipotézis mellett az eddigi
      megfigyelések fényében, de a bizonytalanságot nem tudjuk nullára
      csökkenteni
    \end{itemize}
  \end{itemize}
\end{itemize}

A statisztika a kompromisszomkról szól

\begin{itemize}
\item
  Nincs egyetlen objektív módszer a statisztikában
\item
  Mindig a vizsgált kérdés, az adatgyűjtést és elemzést korlátozó
  tényezők függvényében kell megtalálnunk a legmeggyőzőbb elemzési utat

  \begin{itemize}
  \item
    Ahhoz, hogy a választott elemzési út ténylegesen meggyőző legyen
    transzparensen közölnünk kell analitikus döntéséinket és a mellette
    szóló érveket
  \item
    Ez az alapfeltétele az analitikus megismételhetőségnek: más kutató
    ugyanazokat az adatokat vizsgálva megegyező analatikus úton
    ugyanarra az eredményre érkezik

    \begin{itemize}
    \tightlist
    \item
      Ez sajnos ma a tudományban nem magától értetődő!
    \end{itemize}
  \end{itemize}
\item
  Több valid elemzési út is lehet ugyanannak a kérdésnek a vizsgálatára

  \begin{itemize}
  \item
    A szakmai közösség feladata ezeknek a felülbírálata
  \item
    Lásd: multi analyst kutatások
  \end{itemize}
\item
  A tudomány iteratív módon lépésenként épül fel

  \begin{itemize}
  \item
    Lehetséges, hogy a ma még megbízhatónak számító elemzési utat új
    eredmények fényében elvetjük
  \item
    És ezzel nincs baj!
  \end{itemize}
\end{itemize}

\bookmarksetup{startatroot}

\hypertarget{adat}{%
\chapter{Adat}\label{adat}}

\hypertarget{mi-az-adat}{%
\section{Mi az adat?}\label{mi-az-adat}}

\begin{itemize}
\item
  Az \textbf{adat változók összessége}, amelyek valamilyen
  \textbf{mérés} eredményét rögzítik

  \begin{itemize}
  \item
    \textbf{Az adat, angolul data többesszám, egyesszámban datum}

    \begin{itemize}
    \tightlist
    \item
      \textbf{Tehát data are és nem data is!}
    \end{itemize}
  \end{itemize}
\item
  Sokszor gondolunk az adatra, mint egy adott, önmagában létező dologra.
  Ezzel szemben, az adat egy dolog mérés által létrejött reprezentációja

  \begin{itemize}
  \item
    Tehát a mérés módja befolyásolja az adatok jelentését és minőségét
  \item
    Illetve a mérés befolyásolja az adatokból levont következtetéseket
    is
  \item
    Ismerhetünk akárhány statisztikai trükköt az adatok tisztítására,
    transzformálására, összesítésére, ha azok zajos, torzított, hibás
    mérésből származnak az adatokból statisztika útján levont
    következtetések is torzítani fognak
  \item
    Ezért fontos már a kutatás megtervezésekor nagyon odafigyelni
    \textbf{mit} mérünk és \textbf{milyen pontosságal} tudjuk mérni.
  \item
    Az adatok létrejöttével a kutatásmódszertan foglalkozik
  \item
    Példa:

    \begin{itemize}
    \tightlist
    \item
      1999-ben a Mars Climate Orbiter a Mars atmoszférájába lépve
      darabokra hullott, mert az egyik modulja angolszász
      mértékegységeket használt, míg a másik metrikusat
    \end{itemize}
  \item
    További olvasmányok:

    \begin{itemize}
    \item
      \url{https://www.psychologicalscience.org/observer/measurement-matters}
    \item
      \url{https://journals.sagepub.com/doi/full/10.1177/2515245920952393}
    \end{itemize}
  \end{itemize}
\item
  Minden változónak legalább két különböző értéke kell legyen, hogy
  változónak nevezzük, különben egy konstans lenne
\item
  A változóknak különböző típusai lehetnek, annak fényében milyen mérés
  eredményeit rögzítik
\end{itemize}

\hypertarget{hogyan-keletkeznek-az-adatok}{%
\section{Hogyan keletkeznek az
adatok?}\label{hogyan-keletkeznek-az-adatok}}

\hypertarget{mit-muxe9ruxfcnk}{%
\subsection{Mit mérünk?}\label{mit-muxe9ruxfcnk}}

\begin{itemize}
\item
  Elméleti konstruktumok vagy valós fizikai tulajdonságok

  \begin{itemize}
  \item
    Sok esetben, főleg a viselkedéses közgazdaságtanban nem fizikai
    tulajdonságokat mérünk, hanem közvetlenül meg nem figyelhető
    élméleti \textbf{konstruktumokat}

    \begin{itemize}
    \tightlist
    \item
      Pl: boldogság
    \end{itemize}
  \end{itemize}
\item
  Az elméleti konstruktumok mérésénél fontos, hogy hogyan
  \textbf{operacionalizáljuk} azok mérését

  \begin{itemize}
  \tightlist
  \item
    Boldogság: mosolyok száma egy nap alatt vagy önbevallásos kérdőív
    1-től 7-ig terjedő skáláján adott pontok
  \end{itemize}
\item
  \textbf{Validitás (érvényesség)}

  \begin{itemize}
  \item
    Azt mutatja meg, hogy mennyire mérjük azt, amit valójában mérni
    akarunk
  \item
    Többféle validitás mutató is van

    \begin{itemize}
    \item
      \textbf{Látszatérvényesség}

      \begin{itemize}
      \item
        Első ránézésre úgy tűnik-e, hogy a mérőeszközünk azt méri, amit
        mérni akarunk
      \item
        Pl: vérnyomást a nyelv pirossága mentén szeretném megmérni
        valószínűleg a mérőeszközöm nem érvény adatokat fog adni
      \end{itemize}
    \item
      \textbf{Konstruktum validitás}

      \begin{itemize}
      \item
        Két része van:

        \begin{itemize}
        \item
          \textbf{Konvergens validitás}

          \begin{itemize}
          \item
            A mérőeszköz mérései pozitív együttjárást mutatnak ugyanazt
            a konstruktumot vizsgáló más mérőeszközök méréseivel

            \begin{itemize}
            \tightlist
            \item
              Pl: boldogságot vizsgálom kérdőíves és interjú módszerrel
              is, az eredmények között pozitív kapcsolatot várok el
            \end{itemize}
          \end{itemize}
        \item
          \textbf{Divergens validitás}

          \begin{itemize}
          \tightlist
          \item
            A mérőeszköz mérései és más mérőeszközök mérései között
            valóban nincs kapcsolat, ha elméleti szinten nem
            feltételezünk kapcsolatot a két vizsgált konstruktum között
          \end{itemize}
        \end{itemize}
      \item
        Új mérőeszköz fejlesztése esetén általában már validált, a
        szakmában elfogadott mérőeszközökhöz hasonlítunk
      \end{itemize}
    \item
      \textbf{Prediktív validitás}

      \begin{itemize}
      \item
        A mérőeszközünk mérései alapján előre tudunk jelezni más, a mért
        dologgal összefüggő egyéb kimeneteleket

        \begin{itemize}
        \tightlist
        \item
          Pl: ha feltételezzük, hogy a figyelmetlenség összefügg a
          pénzszórással, akkor azt várjuk, hogy aki a figyelmetlenségi
          kérdőíven magas pontszámot ér el, átlagosan jobban fogja
          szórni a pénzt a hétköznapokban
        \end{itemize}
      \end{itemize}
    \end{itemize}
  \end{itemize}
\end{itemize}

\hypertarget{milyen-pontossuxe1ggal-muxe9ruxfcnk}{%
\subsection{Milyen pontossággal
mérünk?}\label{milyen-pontossuxe1ggal-muxe9ruxfcnk}}

\begin{itemize}
\item
  Minden mérésben van mérési hiba
\item
  Általában ezt akarjuk csökkenteni
\item
  Megtehetjük:

  \begin{itemize}
  \item
    Mérőeszköz minőségének javításával
  \item
    Még több méréssel
  \end{itemize}
\item
  \textbf{Megbízhatóság (reliabilitás)}

  \begin{itemize}
  \item
    Azt mutatja meg, hogy mennyire pontosan, konzisztensen mér a
    merőeszközünk
  \item
    Teszt-retesz megbízhatóság

    \begin{itemize}
    \tightlist
    \item
      Ha többször megismételjük a mérést, mennyire járnak együtt a
      mérések eredményei
    \end{itemize}
  \item
    Megbízhatóság fontos, ha össze akarunk több mérést hasonlítani

    \begin{itemize}
    \item
      Két változó közötti kapcsolat nem lehet erősebb, mint egy változó
      és önmaga közötti kapcsolat (másnéven a változó megbízhatósága)
    \item
      Nem megbízható mérés nem tud erős statisztikai összefüggésben
      lenni egy másik méréssel
    \end{itemize}
  \end{itemize}
\end{itemize}

\hypertarget{a-vuxe1ltozuxf3k-muxe9ruxe9si-szintjei}{%
\subsection{A változók mérési
szintjei}\label{a-vuxe1ltozuxf3k-muxe9ruxe9si-szintjei}}

\begin{itemize}
\item
  Hogyan viszonyulnak egymáshoz a változó értékei
\item
  Folytonos vagy diszkrét változók

  \begin{itemize}
  \item
    \textbf{Folytonos:} bármilyen értéket felvehet egy bizonyos
    értéktartományban
  \item
    \textbf{Diszkrét:} csak bizonyos értékeket vehet fel
  \end{itemize}
\item
  Négy szempont alapján térhetnek el:

  \begin{itemize}
  \item
    \textbf{Egyediség:} a változó minden értéke egyedi jelentéssel
    rendelkezik
  \item
    \textbf{Sorbarendezhetőség:} sorba rendezhetők-e a változó értékei
    valamilyen szempont alapján
  \item
    \textbf{Egyenlő távolság:} az egyes értékek közötti távolság a skála
    minden pontján egyenlő

    \begin{itemize}
    \tightlist
    \item
      1cm és 2cm között ugyanakkora a távolság, mint 12cm és 13cm között
    \end{itemize}
  \item
    \textbf{Abszolút nulla pont:} van-e a skálának abszolút nulla
    pontja.

    \begin{itemize}
    \tightlist
    \item
      Például súly vagy magasság
    \end{itemize}
  \end{itemize}
\item
  Ezek alapján a következő mérési szintekről beszélhetünk:

  \begin{itemize}
  \item
    \textbf{Nominális skála}

    \begin{itemize}
    \item
      A skálának minden tagja egy egyedi értéket jelöl
    \item
      A skálában lévő számok kvalitatív értékek címkézésére vannak
    \item
      Nem rakhatók nagyságrendileg sorrendbe, nincs nulla pontjuk, a
      köztük lévő távolság értelmezhetetlen
    \item
      Matematikai módszerek: összehasonlíthatók
    \item
      Statisztikai módszerek: módusz
    \item
      Diszkrét
    \item
      Legtöbbször az adattáblában egész számokkal vagy szöveggel
      jelöljük

      \begin{itemize}
      \tightlist
      \item
        Az egész számok csak jelző címkeként szolgálnak, algebrai
        műveletek nem végezhetők rajtuk
      \end{itemize}
    \end{itemize}
  \item
    \textbf{Ordinális skála}

    \begin{itemize}
    \item
      Értékei egyediek és sorbarendezhetők
    \item
      De az értékek közötti távolság értelmezhetetlen
    \item
      Pl: fájdalom mérésére szolgáló skála

      \begin{itemize}
      \tightlist
      \item
        A Likert-típusú skálánál például elmondható, hogy a ``Teljesen
        egyetértek'' pozitívabb, mint az ``Inkább egyetértek'', de a
        kettő közti távolság nem kvantifikálható. Ezért nem tudunk
        Likert-típusú skáláknál átlagot számolni.
      \end{itemize}
    \item
      Matematikai módszerek: megnézhetjük, hogy az egyik érték nagyobb-e
      a másiknál
    \item
      Statisztikai módszerek: medián
    \item
      Diszkrét
    \item
      Az adattáblában általában egész számokkal jelöljük őket
    \end{itemize}
  \item
    \textbf{Intervallum skála}

    \begin{itemize}
    \item
      Az ordinális skálánál több, amennyiben a felvehető értékek közötti
      távolság állandó
    \item
      Pl: hőfok Celsiusban
    \item
      Matematikai módszerek: összeadhatjuk vagy kivonhatjuk
    \item
      Statisztikai módszerek: átlag
    \item
      Folytonos
    \item
      Az adattáblában egyész számokkal vagy valós számokkal jelöljük
      őket
    \end{itemize}
  \item
    ~\textbf{Arány skála}

    \begin{itemize}
    \item
      Az interval skálán felül van értelmezhető abszolút 0 pontja.
    \item
      Pl: Súly és magasság
    \item
      Matematikai módszerek: szorozhatjuk vagy oszthatjuk
    \item
      Statisztikai módszerek: átlag
    \item
      Folytonos
    \item
      Az adattáblában egyész számokkal vagy valós számokkal jelöljük
      őket
    \end{itemize}
  \end{itemize}
\end{itemize}

\hypertarget{muxe9ruxe9si-skuxe1luxe1k-hatuxe1sa-a-statisztikai-tesztek-elux151feltuxe9teleire}{%
\subsection{Mérési skálák hatása a statisztikai tesztek
előfeltételeire}\label{muxe9ruxe9si-skuxe1luxe1k-hatuxe1sa-a-statisztikai-tesztek-elux151feltuxe9teleire}}

\begin{itemize}
\item
  A változók mérési szintjei meghatározzák, hogy milyen
  matematikai/statisztikai módszereket alkalmazhatunk rajtuk és így
  végül milyen kérdéseket tudunk általuk megválaszolni
\item
  A változók mérési szintjei együtt járnak a mérési skálák
  \textbf{granularitás}ával

  \begin{itemize}
  \tightlist
  \item
    Granularitás: változó szintjeinek száma, azaz hány lehetséges
    értéket tud felvenni a változó
  \end{itemize}
\item
  Például a fájdalmat mérhetjük úgy, hogy:

  \begin{itemize}
  \item
    Vizuális analóg skálánál a végpontok vannak megadva (nincs fájdalom
    - legerősebb elképzelhető fájdalom) közte 1-100-ig bejelölhetik a
    résztvevők az átélt fájdalmuk
  \item
    Likert-típusú skálán is mérhetjük, aminek három szintje van és
    mindegyik szinthez egy leíró címke tartozik (1 - egyáltalán nem fáj,
    2- közepesen fáj, 3 - nagyon fáj)
  \item
    Numerikus skálán is mérhetjük, aminek 11 szintje (0 - 10) van az
    első ötödik és tizedik szinteken leíró címkével
  \end{itemize}
\item
  Ha egy skála granularitása kicsi (\textless12) vagy minden szintjéhez
  címke tartozik, nem kezelhetjük intervallum skálaként

  \begin{itemize}
  \tightlist
  \item
    Tehát nem számolhatunk átlagot
  \end{itemize}
\item
  Mérési skálák granularitását növelhetjük az itemek számával

  \begin{itemize}
  \item
    Pl: Több 7 fokú válasz skálás kérdéssel mérjük ugyanazt a
    konstruktumot

    \begin{itemize}
    \tightlist
    \item
      A válaszokat összesítjük (például összeadás vagy átlagolás útján)
    \end{itemize}
  \end{itemize}
\end{itemize}

\hypertarget{hogyan-struktuxfaruxe1ljuk-az-adatokat}{%
\section{Hogyan struktúráljuk az
adatokat?}\label{hogyan-struktuxfaruxe1ljuk-az-adatokat}}

\begin{itemize}
\item
  Az adatok különböző formátumban léteznek

  \begin{itemize}
  \tightlist
  \item
    Kérdőíves felmérések adatai, orvosi MRI képek, banki tranzakciók
    lenyomatai, twitter megosztások száma, egy adott időszakban
    megjelent publicisztikák
  \end{itemize}
\item
  A nyers adatok kinyerése és tisztítása teszi ki általában az
  adatelemzői munka nagy részét a statisztikai elemzéshez képest
\item
  Az egyszerűség kedvéért az órán adattáblákkal fogunk dolgozni
\item
  Az \textbf{adattábla} változók és megfigyelések összessége

  \begin{itemize}
  \item
    \textbf{Adatpont (data point)}: egy mérésből származó érték, pl. egy
    ember egy válasza egy kérdésre

    \begin{itemize}
    \tightlist
    \item
      cellák
    \end{itemize}
  \item
    \textbf{Megfigyelés (observation)}: ugyanannak a megfigyelési
    egységnek (pl. egy ember) az összes adatpontja (pl. összes kérdésre
    adott válasza)

    \begin{itemize}
    \tightlist
    \item
      sorok
    \end{itemize}
  \item
    \textbf{Változó (variable)}: több megfigyelési egységnek az
    ugyanarra a mérésre adott értékei

    \begin{itemize}
    \tightlist
    \item
      Oszlopok
    \end{itemize}
  \end{itemize}
\end{itemize}

\hypertarget{milyen-egy-juxf3-adattuxe1bla}{%
\subsection{Milyen egy jó
adattábla?}\label{milyen-egy-juxf3-adattuxe1bla}}

\begin{itemize}
\item
  ``négyzetes'': a sorok és oszlopok következetes módon megfigyeléseket
  és változókat jelentenek
\item
  A változónevek emberi és gépi olvasásra is alkalmasak

  \begin{itemize}
  \tightlist
  \item
    minél rövidebb, a változónevek mindig angolul, ékezetek nélkül,
    szóközök nélkül legyenek írva, legyen egyértelmű jelentésük, pl.
    education, bdi\_1, bdi\_sum
  \end{itemize}
\item
  Változónevek egy táblán belül következetesen legyenek használva

  \begin{itemize}
  \item
    Lásd: snake\_case vagy camelCase
  \item
    Azonos információ egységek azonos helyen legyenek elszeparálva

    \begin{itemize}
    \item
      question\_01\_item\_01, question\_01\_item\_02
    \item
      És nem: question01\_item\_01, question\_01\_item02
    \end{itemize}
  \end{itemize}
\item
  A címkék szó szerint szerepelnek, és nem kódolva (pl. a kategorikus
  adatokhoz)

  \begin{itemize}
  \tightlist
  \item
    ``férfi'', és nem 1)
  \end{itemize}
\item
  A változó egy információt tartalmaz a megfigyelésről

  \begin{itemize}
  \tightlist
  \item
    ``Férfi 18-29'' az két információ
  \end{itemize}
\item
  Minden információ explicit adat (és nem formázási mód vagy komment)
\item
  A hiányzó adat hiányzó cella, nem pedig pl. -999
\item
  Minden megfigyelési egységnek (pl. résztvevő) egyedi azonosítója van
\item
  Van hozzá ``kódkönyv'' (code book vagy data dictionary)

  \begin{itemize}
  \item
    Ez szintén legyen ember és gépi olvasásra alkalmas
  \item
    Információ az adatok formátumáról (pl. adattábla x sorral és y
    változóval)
  \item
    Változók jelentése, mértékegysége, típusa, a változó hogyan lett
    kiszámítva
  \item
    Információ az adatgyűjtés módszeréről
  \end{itemize}
\item
  Változón belül az értékek konzisztensen azonos típusúak és jelölésűek
  legyenek

  \begin{itemize}
  \tightlist
  \item
    Rossz példa: férfi, Férfi, nő, female
  \end{itemize}
\item
  Figyeljünk a kis és nagybetűkre!
\item
  Számozási sorrendnél használjunk vezető 0-át

  \begin{itemize}
  \item
    Pl: 01, 02, 03, 11, 12
  \item
    És nem: 1, 11, 12, 2, 3
  \end{itemize}
\end{itemize}

\hypertarget{hogyan-dolgozzunk-az-adatokkal}{%
\section{Hogyan dolgozzunk az
adatokkal?}\label{hogyan-dolgozzunk-az-adatokkal}}

\begin{itemize}
\item
  Adatrendezés alapelvei

  \begin{itemize}
  \item
    Nem törlünk ki adatokat
  \item
    Adatállomány feldolgozottsági szintjei szerinti felosztás

    \begin{itemize}
    \item
      \textbf{Source:} forrás adatok, ahogy a mérőeszközünk visszaadja
      az adatokat bármilyen beavatkozás nélkül, sokszor tartalmaz olyan
      adatokat, amelyek a résztvevők beazonosítását lehetővé teszik

      \begin{itemize}
      \tightlist
      \item
        a forrás adatok nyílt hozzáférésű megosztása ebben az esetben
        tilos
      \end{itemize}
    \item
      \textbf{Raw:} nyers adatok, lényeges beavatkozás nem történt az
      adatfájlon, változó nevek lettek standardizálva, fájl formátuma
      lett megváltoztatva, a beazonosítást lehetővé tevő változók lettek
      maszkolva vagy kitörölve
    \item
      \textbf{Processed:} feldolgozott adatok, az adatszűrés és
      szükséges transzformációk után
    \end{itemize}
  \item
    Nem módosítjuk az eredeti állományt, legfeljebb a másolato(ka)t
  \item
    Ha lehet, készítsünk adatkezelési tervet (data management plan)

    \begin{itemize}
    \item
      Sok pénzügyi támogató elvárja már!
    \item
      Biztosíthatjuk vele, hogy nem követünk el adatkezelési hibákat
    \item
      Hogyan készítsünk jó adatkezelési tervet?

      \begin{itemize}
      \tightlist
      \item
        \url{https://journals.plos.org/ploscompbiol/article?id=10.1371/journal.pcbi.1004525}
      \end{itemize}
    \end{itemize}
  \end{itemize}
\end{itemize}

\hypertarget{adatkezeluxe9si-hibuxe1k}{%
\section{Adatkezelési hibák}\label{adatkezeluxe9si-hibuxe1k}}

\begin{itemize}
\item
  Általában arról hallunk a médiában, ha egy kutatót adahamisítással
  lepleznek le
\item
  Kevesebb szó esik az adatkezelési hibákról, pedig feltételezhetően
  ezek sokkal gyakrabban fordulnak elő
\item
  Az adatkezelési hibákat sokszor akaratlanul követik el a kutatók
\item
  Az adatkezelési hibáknak lehetnek súlyos következményei

  \begin{itemize}
  \tightlist
  \item
    megváltoztathatják például az adatokből levont köveztetéseket
  \end{itemize}
\item
  Reinhart \& Rogoff (2010) közgazdászok például a kimutatták, hogy egy
  állam valós gazdasági növekedése lelassul, ha az államtartozás a GDP
  90\%-át meghaladja

  \begin{itemize}
  \item
    Eredményeik az USA költségvetési tervébe is bekerült
  \item
    Később kiderült, hogy nem jelölték ki az összes adatsort az elemzés
    során, így a vizsgált adatoknak csak egy része került elemzésre
  \item
    Az újraelemzés az eredeti követeztetéseket invalidálta
  \end{itemize}
\item
  Az adatkezelési hibák nagy része az emberi hibából (human error) fakad

  \begin{itemize}
  \tightlist
  \item
    Kognitív kapacitásaink végesek, elfáradunk, nyomás alatt rosszabbul
    teljesítünk, stb.
  \end{itemize}
\item
  Ezért fontos olyan adatkezelési rendszert kiépíteni, amely ezen hibák
  előfordulási valószínűségét csökkenti
\item
  További olvasmányok:

  \begin{itemize}
  \item
    \url{https://journals.sagepub.com/doi/full/10.1177/25152459211045930}
  \item
    \url{https://www.nber.org/papers/w15639}
  \end{itemize}
\end{itemize}

\bookmarksetup{startatroot}

\hypertarget{leuxedruxf3-statisztika}{%
\chapter{Leíró statisztika}\label{leuxedruxf3-statisztika}}

\begin{itemize}
\item
  Célja a minta összefoglalása

  \begin{itemize}
  \tightlist
  \item
    Gyakoriságok, százalékok, csoport átlagok és szórások, stb. által
  \end{itemize}
\item
  Az adatokkal való munka során az adatok valamilyen formában történő
  \textbf{összesítését} végezzük
\item
  Erre azért van szükség, mert az emberek nem képesek a nyers adatokból
  jól következtetéseket levonni
\item
  A folyamat során elkerülhetetlenül elveszik valamennyi információ, ám
  lehetőségünk nyílik átlátni a részletek mögött rejlő összefüggéséket
  és általánosítani azokból
\item
  Az adatösszesítésnek többféle módja lehet

  \begin{itemize}
  \item
    Ez függ a vizsgált kérdéstől
  \item
    És az adatok milyenségétől (például az egyes változók típusától)
  \end{itemize}
\end{itemize}

\hypertarget{uxe9rtuxe9kek-eloszluxe1sa}{%
\section{Értékek eloszlása}\label{uxe9rtuxe9kek-eloszluxe1sa}}

\begin{itemize}
\item
  Az adatok leírásának egyik módja az \textbf{eloszlás}uk vizsgálata

  \begin{itemize}
  \tightlist
  \item
    Ez alapján meg tudjuk mondani, hogy mely értékek gyakoriak és melyek
    ritkák az adattáblánkban
  \end{itemize}
\item
  Az eloszlás megmondja, hogyan oszlik meg az adat a különböző
  lehetséges értékek között

  \begin{itemize}
  \tightlist
  \item
    Például: diszkrét változó esetén hányan választották az egyes
    lehetséges opciókat
  \end{itemize}
\item
  Az egyik gyakran vizsgált eloszlás fajta a \textbf{gyakorisági
  eloszlás}

  \begin{itemize}
  \item
    A gyakorisági eloszlás megmondja, hogy egy változó lehetséges
    értékei milyen gyakran fordulnak elő a mintánkban
  \item
    Diszkrét változó gyakorisági eloszlása

    \begin{itemize}
    \item
      Példa

      \begin{itemize}
      \tightlist
      \item
        Példa\ldots{}
      \end{itemize}
    \item
      Ez a táblázat az adatok abszolút frekvenciáját mutatja meg
    \item
      Ahhoz, hogy meg tudjuk mondani a gyakoriságok között különbség
      valóban lényeges-e hasznos lehet megtekinteni a relatív
      frekvenciájukat (sűrűségüket)
    \item
      Ehhez minden lehetséges értékhez el kell osztani az abszolút
      gyakoriságot a változóban található összes érték számával
    \end{itemize}
  \item
    Hogyan ábrázoljuk a gyakoriságot?

    \begin{itemize}
    \item
      Dotplot

      \begin{itemize}
      \item
        Minden értéket egy pötty jelöl
      \item
        Az x tengely az értéktartomány
      \item
        Az y tengely a gyakoriság
      \item
        Folytonos változónál az összesítés során az értéktartományt
        különböző méretű ``vödrökre'' osztjuk
      \item
        Meghatározzuk a vödrök szélességét (binwidth)
      \item
        És az ebbe az értéktartományba eső értékek gyakoriságát
        számoljuk meg és ábrázoljuk
      \end{itemize}
    \item
      A hisztogram hasonló, mint a dotplot

      \begin{itemize}
      \tightlist
      \item
        De az egyes értékek nem jelennek meg külön, így sok adat
        gyakoriságát egyszerre tudjuk ábrázolni
      \end{itemize}
    \end{itemize}
  \end{itemize}
\item
  \textbf{Kumulatív gyakorisági eloszlás}

  \begin{itemize}
  \item
    Azon értékek gyakorisága amelyek akkor vagy kevesebbek, mint az
    adott határérték, amit vizsgálunk
  \item
    Kiszámolásához összeadjuk a hátárétrék gyakoriságát és az összes
    nála kisebb érték gyakoriságát
  \item
    Értéke soha nem csökkenhet
  \item
    példa
  \end{itemize}
\end{itemize}

\hypertarget{statisztikai-modellek}{%
\section{Statisztikai modellek}\label{statisztikai-modellek}}

\begin{itemize}
\item
  Leegyszerüsített reprezentációja az adatoknak
\item
  Leírja az adatok struktúráját
\item
  Mindig van benne hiba faktor, kihagy részleteket
\item
  \emph{``All models are wrong but some are useful''} (George Box)
\item
  A statisztikai modellre gondolhatunk úgy is, mint egy elméletre, amely
  leírja hogyan keletkeztek az adatok
\item
  Célunk: olyan modellt alkossunk, ami hatékonyan és pontosan írja le az
  adatok keletkezésének a módját

  \begin{itemize}
  \item
    Másszóval, fontos, hogy a modell jól illeszkedjen az adatokra

    \begin{itemize}
    \tightlist
    \item
      Minél jobban illeszkedik a modell az adatokhoz (a valósághoz),
      annál megbízhatóbbak a modell által létrehozott predikciók
    \end{itemize}
  \end{itemize}
\end{itemize}

Mivel minden modell a valóság leegyszerüsítése, így az soha nem fogja
tökéletesen reprezentálni a valóságot

\begin{itemize}
\item
  lesz hiba a modell által prediktált értékek és az egyes megfigyelések
  között
\item
  Adat = modell + hiba

  \begin{itemize}
  \item
    \textbf{Modell:} az értékek, amit az elméletünk alapján elvárunk
  \item
    \textbf{Hiba:} a modell által prediktált értékek és a valós
    adatpontok közötti különbség
  \end{itemize}
\end{itemize}

\hypertarget{a-legegyszerux171bb-statisztika-modell-az-uxe1tlag}{%
\subsection{A legegyszerűbb statisztika modell, az
átlag}\label{a-legegyszerux171bb-statisztika-modell-az-uxe1tlag}}

\begin{itemize}
\item
  Minden megfigyelésre ugyanazt az értéket prediktálja
\item
  Példa

  \begin{itemize}
  \item
    Vizsgálhatjuk például, hogy a statisztika tanároknak hány barátja
    van
  \item
    Tegyük fel, hogy 5 tanárt kérdezünk meg
  \item
    Az átlag: (1 + 2 + 3 + 3 + 4)/5 = 2.6
  \item
    Ebből is látszik, hogy az átlag egy statisztikai modell, hiszen egy
    hipotetikus érték, nem megfigyelhető a valóságban
  \end{itemize}
\end{itemize}

\hypertarget{modell-illeszkeduxe9suxe9nek-a-vizsguxe1lata}{%
\subsection{Modell illeszkedésének a
vizsgálata}\label{modell-illeszkeduxe9suxe9nek-a-vizsguxe1lata}}

\begin{itemize}
\item
  Hogyan tudjuk megvizsgálni, hogy mennyire jól illeszkedik az átlag,
  mint modell, az adatokra?

  \begin{itemize}
  \item
    Ahhoz, hogy eldöntsük a modell jól írja-e le az adatokat,
    megnézhetjük a megfigyelt adatok és a modell értékei közötti
    különbséget
  \item
    Ezen értékek közötti eltérés a modell hibája
  \item
    Például annál a kutatónál, akinek egy barátja van, a modell 2.6
    barátot prediktál (ez volt az átlag), a modell hibája, azaz az
    eltérés, így megfigyelt értél - prediktált érték = -1.6
  \item
    Ebben az esetben a modell felül becsüli a kutató népszerűségét!
  \item
    Hogyan összesítsük az egyes eltéréseket, hogy meg tudjuk határozni a
    modellünk pontosságát?

    \begin{itemize}
    \item
      Például \textbf{összeadhatjuk őket (sum of errors)}

      \begin{itemize}
      \item
        Ebben az esetben azt látjuk, hogy az eltérések összege 0
      \item
        Ez alapján arra következtethetnénk, hogy az átlag tökéletesen
        reprezentálja az adatokat, azonban az ábrára ránézve láthatjuk,
        hogy ez nem így van

        \begin{itemize}
        \item
          az egyes értékek és a modell által prediktált értékek között
          van különbség

          \begin{itemize}
          \tightlist
          \item
            Még egy ok az ábrázolás fontossága mellett!
          \end{itemize}
        \end{itemize}
      \item
        A negatív és a pozitív előjelű eltérések kiegyenlítették egymást
      \end{itemize}
    \item
      Ezt elkerülendő \textbf{négyzetre emelhetjük az eltéréseket az
      összeadás előtt (sum of squared errors)}

      \begin{itemize}
      \item
        Így minden eltérés előjele pozitív lesz
      \item
        A példánkban a négyzetes eltérések összege 5.20
      \item
        Most azonban abba a problémába ütközünk, hogy a modell
        pontosságának mérője függ a mintánk méretétől
      \item
        Minél több megfigyelésünk van, annál nagyobb lesz a modell hiba
      \end{itemize}
    \item
      Ezt elkerülhetjük úgy, hogy összeadás helyett a \textbf{négyzetes
      eltérések átlagát vesszük (mean of squared errors)}

      \begin{itemize}
      \item
        Ehhez elosztjuk a négyzetes eltérések összegét a megfigyelések
        számával
      \item
        Így azonban csak a mintánkban lévő átlagos hibát számszerüsítjük
      \item
        Azonban célunk, hogy a mintában lévő hibából a populációban
        található hibát becsüljük meg
      \item
        Ehhez a minta mérete helyett a szabadságfokkal kell elosztanunk
        a négyzetes eltérések összegét
      \item
        Jelen esetben ez n - 1
      \end{itemize}
    \item
      Ezt nevezzük \textbf{varianciának}

      \begin{itemize}
      \item
        Átlagos négyzetes eltérés az átlagtól
      \item
        A varianciával a probléma, hogy a mértékegysége az adatok
        skálájának négyzete
      \item
        A példánkban 1.3

        \begin{itemize}
        \tightlist
        \item
          Négyzetes barátok száma nehezen értelmezhető!
        \end{itemize}
      \end{itemize}
    \item
      Ezt elkerülhetjük úgy, hogy a variancia gyökét vesszük, ez a
      \textbf{szórás}

      \begin{itemize}
      \item
        Így a modell hibájának mérője ugyanazt a skálát használja, mint
        a megfigyeléseink
      \item
        Az átlaghoz képest kis szórás azt mutatja, hogy a modellunk jól
        illeszkedik az adatokra

        \begin{itemize}
        \tightlist
        \item
          A megfigyelések közel vannak az átlaghoz
        \end{itemize}
      \item
        Példa

        \begin{itemize}
        \item
          Megkérhetjük a hallgatókat, hogy egy 5-ös skálán értékeljék az
          egyes statisztika oktatókat
        \item
          A mérést elvégezhetjük 5 egymást követő órán
        \item
          Az ábrán két oktatónak öt óráján mért összesített értékelései
          láthatók
        \item
          Ha az átlagot használjuk, mint statisztikai modellt, mind a
          két oktatónál ugyanazt az átlagot kapjuk
        \item
          Mégis látható, hogy az egyik oktatónál a modell jobban
          illeszkedik az adatokhoz, tehát az átlag pontos
          reprezentációja az adatoknak

          \begin{itemize}
          \tightlist
          \item
            Ezt mutatja, hogy a szórás 0.55 az átlaghoz mérten kicsi
          \end{itemize}
        \end{itemize}
      \end{itemize}
    \end{itemize}
  \end{itemize}
\end{itemize}

\hypertarget{mitux151l-juxf3-egy-statisztikai-modell}{%
\subsection{Mitől jó egy statisztikai
modell?}\label{mitux151l-juxf3-egy-statisztikai-modell}}

\begin{itemize}
\item
  1) kicsi a hiba

  \begin{itemize}
  \item
    Mitől lehet nagy a hiba?

    \begin{itemize}
    \item
      Rossz a modell

      \begin{itemize}
      \item
        Kimaradt egy fontos prediktor változó
      \item
        A prediktor változó hatásának irányát rosszul adtuk meg

        \begin{itemize}
        \tightlist
        \item
          Pl: azt feltételezi a modell minél idősebb valaki annál
          alacsonyabb gyerekek körében
        \end{itemize}
      \item
        Az adatok vizualizációja fontos, hogy jól specifikált modellt
        tudjunk építeni
      \end{itemize}
    \item
      Mérés hiba/zaj/adatokban lévő variancia miatt

      \begin{itemize}
      \item
        Vagy a mérőeszköz nem elég pontos
      \item
        Vagy egyébb faktorok is befolyásolják a megfigyelt mérésekben
        lévő varianciát, amikről nem tudunk vagy nem tudjuk mérni
      \end{itemize}
    \end{itemize}
  \end{itemize}
\item
  2) jól generalizálható

  \begin{itemize}
  \tightlist
  \item
    ha új adatokra illesztjük a modellt, azokat is jól fogja prediktálni
  \end{itemize}
\end{itemize}

\hypertarget{uxf6sszesuxedtux151-statisztikuxe1k}{%
\section{Összesítő
statisztikák}\label{uxf6sszesuxedtux151-statisztikuxe1k}}

\begin{itemize}
\item
  \textbf{Gyakoriság (frequency):} a megfigyelések száma (db). Pl. az
  előadást megnéző hallgatók száma.
\item
  \textbf{Összeg (summary):} egy változó összes értékének összeadásával
  keletkező érték. Pl. covid megbetegedések száma.
\item
  \textbf{Arány (proportion):} a megfigyelések száma az összes
  megfigyeléshez képest (Pl. 54 \% vagy 0.54 vagy .54). pl. biciklisek
  aránya az összes közlekedőhöz képest az Andrássy úton
\end{itemize}

\hypertarget{kuxf6zuxe9puxe9rtuxe9kek}{%
\subsection{Középértékek}\label{kuxf6zuxe9puxe9rtuxe9kek}}

\begin{itemize}
\item
  \textbf{Matematikai átlag (mean, average):} Az értékek összege
  elosztva az értékek számával

  \begin{itemize}
  \item
    Általában folytonos változóknál vagy nagy granilaritással rendelkező
    ordinális változónál használjuk
  \item
    Akkor jó használni, ha a változónk eloszlása szimmetrikus
  \item
    Ha nagy kiugró értékek vannak az átlag torzíthat

    \begin{itemize}
    \item
      Az átlag a négyzetes hibák összegét csökkenti
    \item
      A hibák négyzetre emelésénél a kiugró értékeknél exponenciálisan
      nő a hiba
    \end{itemize}
  \end{itemize}
\item
  \textbf{Medián (median):} A nagyság szerint sorba rendezett értékek
  közül a középső. Ha páros számú érték van, akkor általában a középső
  kettő átlaga

  \begin{itemize}
  \item
    Ordinális változónál használjuk
  \item
    Akkor jó használni, ha ferde az eloszlás vagy vannak outlierek

    \begin{itemize}
    \item
      A hibák abszolút értékének összegét csökkenti
    \item
      Ezért kevésbé érzékeny, nincs négyzetre emelés
    \end{itemize}
  \end{itemize}
\item
  \textbf{Módusz (mode):} A leggyakrabban előforduló érték

  \begin{itemize}
  \tightlist
  \item
    Nominális változónál használjuk
  \end{itemize}
\end{itemize}

\hypertarget{szuxe9lsux151uxe9rtuxe9kek}{%
\subsection{Szélsőértékek}\label{szuxe9lsux151uxe9rtuxe9kek}}

\begin{itemize}
\item
  \textbf{Minimum:} a legkisebb érték
\item
  \textbf{Maximum:} a legnagyobb érték
\item
  \textbf{Kiugró értékek (outlier):} olyan érték, ami a többitől távol
  esik

  \begin{itemize}
  \item
    Azt, hogy milyen vágási ponttól számít egy érték outliernek sokszor
    nem könnyen határozható meg.
  \item
    Függhet az elmélettől vagy az adott szakterületen használt
    konvenciók is megszabhatják
  \item
    Az outlierek nagy torzító hatással vannak az átlagra

    \begin{itemize}
    \item
      Érdemes ezért megvizsgálni a változónk eloszlását, mielőtt úgy
      döntünk, hogy az átlag alapján kívánjuk összesíteni az adatainkat

      \begin{itemize}
      \tightlist
      \item
        Lehet, hogy torzítani fog = nem jól reprezentálja az adatokat
      \end{itemize}
    \end{itemize}
  \item
    Kevésbé torzítják a mediánt
  \item
    Nem torzítják a móduszt
  \end{itemize}
\end{itemize}

\hypertarget{helyzetmutatuxf3k}{%
\subsection{Helyzetmutatók}\label{helyzetmutatuxf3k}}

\begin{itemize}
\item
  Kvantilisek: vágási pontok, amelyek mentén a sorba rendezett adatokat
  meghatározott számú részre bonthatjuk
\item
  Pl. kvartilisek: Az adatokat négy egyenlő részre osztó három pont (ld.
  még decilis, percentilis)
\item
  Interkvartilis tartomány (IQR): A felső (75\%) alsó (25\%) kvartilis
  és az alsó kvartilis különbsége, az adatok középső 50\%-a~

  \begin{itemize}
  \tightlist
  \item
    Általában ordinális adatok összesítésénél szoktuk használni
  \end{itemize}
\end{itemize}

\bookmarksetup{startatroot}

\hypertarget{adatvizualizuxe1ciuxf3}{%
\chapter{Adatvizualizáció}\label{adatvizualizuxe1ciuxf3}}

\begin{itemize}
\item
  Példa adatvizualizáció fontosságára: John Snow és a londoni kolera
  járvány
\item
  Szerepe:

  \begin{itemize}
  \item
    Az adatok ellenőrzése

    \begin{itemize}
    \item
      Hasonló középértékekkel és szórásokkal rendelkező nyers adatok
      nagyon más értékeket vehetnek fel
    \item
      Segít megtalálni hibákat is az adatokban
    \end{itemize}
  \item
    Egy történet elmesélése

    \begin{itemize}
    \tightlist
    \item
      az adatok közti összefüggés feltárása
    \end{itemize}
  \end{itemize}
\item
  DE a vizualizáció nem szabad megtévesztő legyen!
\end{itemize}

\hypertarget{a-juxf3-adatvizualizuxe1ciuxf3-ismuxe9rvei}{%
\section{A jó adatvizualizáció
ismérvei}\label{a-juxf3-adatvizualizuxe1ciuxf3-ismuxe9rvei}}

\begin{itemize}
\item
  Az ábra minden részének legyen információ értéke

  \begin{itemize}
  \item
    Felesleges vagy redundáns elemek ne kerüljenek rá, mert elterelik a
    figyelmet a lényegről és megzavarják az értelmezést
  \item
    Olyan elemek jelenjenek meg csak az ábrán, amelyek nem törölhetők
    információ veszteség nélkül
  \item
    Lásd:

    \begin{itemize}
    \item
      adat/tinta arány
    \item
      Ábra szemét
    \end{itemize}
  \end{itemize}
\item
  Legjobb megmutatni az egyéni adatpontokat is amennyiben ez lehetséges

  \begin{itemize}
  \tightlist
  \item
    Az összesítés torzíthatja az eredményeket
  \end{itemize}
\item
  Használd ki az ábránál a teret, de figyelj rá, hogy közben ne torzítsd
  az adatokat

  \begin{itemize}
  \item
    Különböző ábrázolási módok nagyon más történetet tudnak elmesélni
  \item
    Példa: szerepeljen-e a nulla érték az y tengelyen vagy ne
  \end{itemize}
\item
  Figyelj az emberi percepció limitációira

  \begin{itemize}
  \item
    Színvakság

    \begin{itemize}
    \tightlist
    \item
      Használj színvak barát színeket
    \end{itemize}
  \item
    Pie chart

    \begin{itemize}
    \item
      Az embereknek ennél a vizualizációs módszernél nagyon nehéz az
      arányokat helyesen értelmezni!
    \item
      Ne használjuk!
    \end{itemize}
  \end{itemize}
\end{itemize}

\hypertarget{gyakori-adatvizualizuxe1ciuxf3s-megolduxe1sok}{%
\section{Gyakori adatvizualizációs
megoldások}\label{gyakori-adatvizualizuxe1ciuxf3s-megolduxe1sok}}

\hypertarget{doboz-uxe1bra-boxplot}{%
\subsection{Doboz ábra (boxplot)}\label{doboz-uxe1bra-boxplot}}

\begin{itemize}
\item
  A helyzetmutatókat általában egy \textbf{boxplot} segítségével
  vizualizáljuk
\item
  Ahol a doboz felső határsa a 75-ik percentilist, az alsó határa a
  25-ik percentilist jelöli

  \begin{itemize}
  \tightlist
  \item
    A két vonal közötti rész az IQR
  \end{itemize}
\item
  A kettő közti vonal a doboz felénél az 50-ik percentilist jelöli

  \begin{itemize}
  \tightlist
  \item
    Másnéven medián
  \end{itemize}
\item
  A dobozból kijövő függőleges vonalak a 75-ik és a 25-ik
  percentileseken kívül eső, de még nem outlier értékeket mutatják

  \begin{itemize}
  \tightlist
  \item
    Ezen kívül pöttyökkel tudjuk jelölni azokat az egyes adatpontokat,
    amelyek outlier értéknek számítanak
  \end{itemize}
\end{itemize}

\hypertarget{hogyan-szuxfarjunk-ki-megtuxe9vesztux151-vizualalizuxe1ciuxf3s-elemeket}{%
\section{Hogyan szúrjunk ki megtévesztő vizualalizációs
elemeket?}\label{hogyan-szuxfarjunk-ki-megtuxe9vesztux151-vizualalizuxe1ciuxf3s-elemeket}}

\begin{itemize}
\item
  Segít-e az ábra megérteni az adatokat, vagy inkább csak összezavar?
\item
  Figyelmesen nézd meg az ábrán a tengelyek nevét és léptékét!
\item
  Nézd meg, hogy a tengelyek a nulláról indulnak-e!~
\item
  Nézd meg, hogy a különböző csoportokat bemutató ábrázolások egyenlő
  arányban változnak-e egymással
\item
  Nem hagytak-e ki adatpontot az ábráról?
\item
  Összevontak-e kategóriákat indokolatlanul?
\item
  Szerepelnek-e olyan adatcsoportok az ábrán, amelyek önkényesen lettek
  kiválasztva?
\item
  Példák rossz ábrákra

  \begin{itemize}
  \tightlist
  \item
    \url{https://venngage.com/blog/misleading-graphs/}
  \end{itemize}
\end{itemize}

\bookmarksetup{startatroot}

\hypertarget{kuxf6vetkeztetuxe9ses-statisztika}{%
\chapter{Következtetéses
statisztika}\label{kuxf6vetkeztetuxe9ses-statisztika}}

\begin{itemize}
\item
  Sokszor amikor egy kérdést vizsgálunk nem tudjuk az egész populációt
  megmérni

  \begin{itemize}
  \tightlist
  \item
    \textbf{Populáció:} egy meghatározott csoport összes lehetséges
    tagja
  \end{itemize}
\item
  A statisztikát használjuk arra, hogy egy minta alapján
  következtetéseket vonjunk le az egész populációra

  \begin{itemize}
  \tightlist
  \item
    \textbf{Minta:} a populációból kiválasztott és megfigyelt egyedek
    részhalmaza
  \end{itemize}
\end{itemize}

\hypertarget{mintavuxe9telezuxe9s}{%
\section{Mintavételezés}\label{mintavuxe9telezuxe9s}}

\begin{itemize}
\item
  A minta kiválasztásának módja fontos, mert ezzel tudjuk biztosítani,
  hogy a minta jól reprezentálja a populációnkat

  \begin{itemize}
  \tightlist
  \item
    valószínűségi mintavételezés: minta jól reprezentálja a populációt,
    ha a populáció minden tagjának egyenlő esélye van a mintánkba való
    bekerülésre
  \item
    kvóta: a populáció néhány ismert jellemzője alapján válogatunk be
    meghatározott számú résztvevőt~ (pl. 50-50\% nő és férfi)
  \item
    kényelmi: az vesz részt a kutatásban, akit éppen elérünk
  \end{itemize}
\item
  A mintavételezés előnyei és nehézségei

  \begin{itemize}
  \item
    előnyei

    \begin{itemize}
    \item
      máshogy nem tudunk információhoz jutni az egész populációra
      vonatkozóan
    \item
      tudunk becsléseket tenni~ a populáció jellemzőire (``valódi''
      átlag, szórás, eloszlás, stb.)
    \item
      ha ismerjük a hátulütőket, tudjuk kommunikálni a bizonytalanságot
      is
    \end{itemize}
  \item
    nehézségei

    \begin{itemize}
    \item
      a minta soha nem tökéletes reprezentációja a populációnak, azaz
      mindig valamennyire torzított
    \item
      a kisebb minták könnyebben torzítottak, mint a nagyobbak
    \item
      nem lehet pontosan tudni, hogy a minta elég jól reprezentálja-e a
      populációt
    \item
      lehet, hogy a minta alapján téves következtetésre jutunk a
      populációra vonatkozóan
    \end{itemize}
  \end{itemize}
\end{itemize}

\hypertarget{standard-hiba}{%
\subsection{Standard hiba}\label{standard-hiba}}

\begin{itemize}
\item
  Attól függetlenül, hogy milyen nagy elővigyázatossággal választottuk
  ki a mintánkat, a minta különbözni fog a populációtól

  \begin{itemize}
  \item
    Ez a \textbf{mintavételezési hiba (sampling variation)}

    \begin{itemize}
    \tightlist
    \item
      Abból következik, hogy a populáció összes egyedéből
      véletlenszerűen választunk ki egyedeket a mintánkba
    \end{itemize}
  \end{itemize}
\item
  Ha többször veszünk mintát egy adott populációból hosszútávon
  elvárható, hogy a mintáink átlagai leggyakrabban a populáció átlag
  körül fognak csoportosulni

  \begin{itemize}
  \item
    A minta átlagok eloszlását a \textbf{mintavételi eloszlásnak
    (sampling distribution)} hívjuk

    \begin{itemize}
    \item
      A mintaátlagok gyakorisági eloszlása
    \item
      Ha a minta átlagok átlagát vennénk a populáció átlagot kapnánk meg
    \end{itemize}
  \item
    A minta átlagok populáció átlag körüli szóródását a standard hibával
    írjuk le

    \begin{itemize}
    \tightlist
    \item
      Máshogyan: Azt a bizonytalanságot fejezi ki, amiben a
      populációátlagtól eltérhet a mintaátlagtól
    \end{itemize}
  \item
    Pontosabb nevén az \textbf{átlag standard hibája (standard error of
    the mean)}.
  \item
    Kiszámolásához a minta szórását elosztjuk az elemszám
    négyzetgyökével
  \item
    A standard hiba alapján a mérésünk minősége a populációban található
    variabilitástól és a mintánk méretétől függ

    \begin{itemize}
    \item
      Mivel csak a minta méretre van ráhatásunk ennek növelésével
      javíthatjuk a mérőeszközünk pontosságát
    \item
      Minél nagyobb a mintaméret, annál biztosabbak lehetünk abban, hogy
      a populációátlagot jól közelítjük

      \begin{itemize}
      \item
        Azonban nem csak a minta mérete számít, hanem a mintavételezés
        módja is

        \begin{itemize}
        \item
          Akármilyen nagy a mintánk, ha annak tagjai nem jól írják le a
          populációt mert a minta szisztematikusan torzít
        \item
          Ezért szoktak kutatók általában random mintavételezésre
          törekedni
        \end{itemize}
      \end{itemize}
    \end{itemize}
  \item
    Általában ezt szokták az ábrákon megjeleníteni \textbf{hibasávként
    (error bar)}
  \item
    Ha a hibasávok nem fednek át, akkor arra következtethetünk, hogy a
    populációban lévő különbség valódi
  \end{itemize}
\end{itemize}

\hypertarget{centruxe1lis-hatuxe1reloszluxe1s-elve}{%
\subsection{Centrális határeloszlás
elve}\label{centruxe1lis-hatuxe1reloszluxe1s-elve}}

\begin{itemize}
\item
  \textbf{Central limit theorem (CHE)}
\item
  A mintaméret növekedésével a mintaátlagok eloszlása közelít a normális
  eloszláshoz.
\item
  Ez akkor is igaz, ha az egyes mintákban lévő eloszlások értéke nem
  normális!
\item
  Demonstráció: \url{https://istats.shinyapps.io/sampdist_cont/}
\item
  A CHE miatt használhatjuk a legtöbb statisztikai módszert ami normális
  eloszlást feltételez
\end{itemize}

\hypertarget{konfidencia-intervallum}{%
\section{Konfidencia intervallum}\label{konfidencia-intervallum}}

\begin{itemize}
\item
  Ahogy már korábban említettük a minta átlagát használjuk arra, hogy
  megbecsüljük a populáció átlagot

  \begin{itemize}
  \tightlist
  \item
    Ezzel az adatokból levont következtetést \textbf{általánosítsuk
    (generalization)} a mintánkon túlra
  \end{itemize}
\item
  Láttuk, hogy különböző mintavételelezések különböző minta átlagokat
  adnak és a standard hibát használhatjuk arra, hogy meghatározzuk
  mekkora a mintaátlagok varianciája a populáció átlag körül
\item
  Azt, hogy a minta átlag mennyire jól becsüli meg a populáció értéket
  úgy is eldönthetjük, hogy kiválasztunk egy értéktartományt, amelybe a
  populáció átlag feltehetően beleesik

  \begin{itemize}
  \tightlist
  \item
    Ezt az értéktartományt nevezzük \textbf{konfidencia intervallum}nak
  \end{itemize}
\item
  Minél szélesebb a konfidencia intervallumunk annál bizonytalanabbak
  vagyunk abban, hogy a minta átlag jó reprezentációja-e a populáció
  átlagnak
\end{itemize}

\hypertarget{helyes-uxe9rtelmezuxe9se}{%
\subsection{Helyes értelmezése}\label{helyes-uxe9rtelmezuxe9se}}

\begin{itemize}
\item
  A konfidencia intervallumot gyakran félreértik a kutatók

  \begin{itemize}
  \tightlist
  \item
    Talán még a p értéknél is gyakrabban!
  \end{itemize}
\item
  Egy 95\%-os konfidencia intervallum nem azt jelenti, hogy 95\% az
  esélye annak, hogy a populáció átlaga beleesik-e az intervallumba!

  \begin{itemize}
  \item
    A populáció átlag egy fix érték, így egyes esetekben vagy beleesik
    vagy nem
  \item
    Nem tudunk valószínűséget rendelni mellé
  \end{itemize}
\item
  A helyes értelmezésnél ugyanazt a logikát kell követnünk, mint a
  hipotézis tesztelésnél: hosszútávon milyen valószínűséggel fogunk
  helyes döntést hozni

  \begin{itemize}
  \item
    hosszútávon a konfidencia intervallum az esetek 95\%-ában fogja
    tartalmazni a populáció átlagot~
  \item
    ha végtelenszer megismételjük a mintavételt és kiszámoljuk a minta
    átlagot és a hozzá tartozó konfidencia intervallumot, akkor az
    esetek 95\%-ában a kapott konfidencia intervallumok magukba fogják
    foglalni a populáció átlagot
  \item
    Azt azonban nem tudhatjuk, hogy az éppen általunk gyűjtött mintához
    kiszámolt konfidencia intervallum tartalmazza-e a populáció átlagot
    vagy sem
  \item
    Másszóval: 95\%-os konfidencia intervallum mellett az esetek 5\%-ban
    tévedünk, ha feltételezzük, hogy az adott konfidencia intervallum
    valóban magába foglalja a populáció átlagot
  \end{itemize}
\end{itemize}

\hypertarget{kiszuxe1muxedtuxe1sa}{%
\subsection{Kiszámítása}\label{kiszuxe1muxedtuxe1sa}}

\begin{itemize}
\item
  A konfidencia intervallumot a z értékek segítségével számoljuk ki
\item
  Egy normál eloszlásnál ahol az átlag 0 és a szórás 1 a z értékek
  95\%-a a -1.96 és a +1.96-os z értékek közé fog esni
\item
  Alsó határa a konfidencia intervallumnak: minta átlag - (1.96 *
  standard hiba)
\item
  Felső határa a konfidencia intervallumnak: minta átlag + (1.96 *
  standard hiba)
\item
  A mintánk átlaga mindig a konfidencia intervallum közepe
\item
  A konfidencia intervallum mérete függ a standard hiba méretétől
\item
  Ha kicsi a minta a t eloszlást használva kell kiszámítani a
  konfidencia intervallumot
\end{itemize}

\hypertarget{vizuuxe1lis-uxe1bruxe1zoluxe1sa}{%
\subsection{Vizuális ábrázolása}\label{vizuuxe1lis-uxe1bruxe1zoluxe1sa}}

\begin{itemize}
\item
  Konfidencia intervallumok vizuális ábrázolása

  \begin{itemize}
  \item
    Általában a konfidencia intervallumot használjuk ábrákon az átlag
    körüli hiba mértékének vizualizására
  \item
    Ha két átlaghoz tartozó konfidencia intervallumok átfednek, akkor
    feltételezhetjük, hogy a két minta átlaga ugyanabból a populációból
    származik

    \begin{itemize}
    \tightlist
    \item
      Lásd t-próba
    \end{itemize}
  \item
    Ha nem fednek át

    \begin{itemize}
    \item
      1) vagy különböző populációból származnak
    \item
      2) vagy ugyanabból a populációból származnak de az egyik
      konfidencia intervallum nem tartalmazza a populáció átlagot

      \begin{itemize}
      \tightlist
      \item
        Ez az esetek 5\%-ában fordul csak elő 95\%-os konfidencia
        intervallum mellett, ezért valószínűleg a 1) opció mellett
        döntünk
      \end{itemize}
    \end{itemize}
  \end{itemize}
\end{itemize}

\bookmarksetup{startatroot}

\hypertarget{null-szignifikancia-hipotuxe9zis-teszteluxe9s}{%
\chapter{Null szignifikancia hipotézis
tesztelés}\label{null-szignifikancia-hipotuxe9zis-teszteluxe9s}}

\begin{itemize}
\item
  Következtetéses statisztika célja a populációra való következtetés a
  minta alapján
\item
  Több statisztikai hipotézis tesztelő keretrendszer is létezik

  \begin{itemize}
  \item
    \textbf{frekventista:}

    \begin{itemize}
    \item
      A hipotézis állandó (vagy igaz vagy nem) és az adatok
      véletlenszerűek
    \item
      Frekventista inferencia az adatok előfordulásának valószínűségére
      fókuszál a hipotézis fényében
    \item
      A frekventista módszerek közül legtöbbször a \textbf{null
      hipotézis szignifikancia tesztelést (NHST)} használják a kutatók
    \end{itemize}
  \item
    \textbf{Bayesiánus:}

    \begin{itemize}
    \item
      a hipotézis valószínűségéről is el tud mondani valamit az előzetes
      tudásunk és az adatok fényében
    \item
      gyakran használt hipotézis tesztelő módszere a \textbf{Bayes
      factor}

      \begin{itemize}
      \item
        két hipotézis valószínűségének az arányát fejezi ki az adatok
        fényében
      \item
        az érdeklődőknek további olvasmányok:

        \begin{itemize}
        \item
          \url{https://alexanderetz.com/2015/08/09/understanding-bayes-visualization-of-bf/}
        \item
          \url{https://www.sciencedirect.com/science/article/abs/pii/S0022249615000607}
        \item
          \url{https://link.springer.com/article/10.3758/s13423-017-1343-3}
        \end{itemize}
      \end{itemize}
    \end{itemize}
  \end{itemize}
\end{itemize}

\hypertarget{null-hipotuxe9zis-szignifikancia-teszteluxe9s}{%
\section{Null hipotézis szignifikancia
tesztelés}\label{null-hipotuxe9zis-szignifikancia-teszteluxe9s}}

\hypertarget{luxe9puxe9sei}{%
\subsection{Lépései}\label{luxe9puxe9sei}}

\begin{itemize}
\item
  Megfogalmazunk egy \textbf{alternatív hipotézist}

  \begin{itemize}
  \item
    A hipotézisünknek lehet iránya \textbf{(egyoldalú hipotézis
    tesztelés)}

    \begin{itemize}
    \item
      Pozitív vagy negatív kapcsolatot várunk el
    \item
      Egy oldalú hipotézis tesztelésnek is nevezik
    \end{itemize}
  \item
    Vagy lehet \textbf{kétoldalú hipotézis teszt}

    \begin{itemize}
    \tightlist
    \item
      Ebben az esetben az alternatív hipotézisünk csak azt várja el,
      hogy a lesz lesz különbség a vizsgált változók között, de az nem
      mondjuk meg milyen irányú különbséget várunk el
    \end{itemize}
  \item
    Azt, hogy melyik hipotézist alkalmazzuk az elmélet (előzetes
    tudásunk) határozza meg
  \end{itemize}
\item
  Megfogalmazzuk a \textbf{null hipotézist}

  \begin{itemize}
  \tightlist
  \item
    Egyoldalú hipotézis esetén a null hipotézis feltételezi, hogy az
    alternatív által prediktált iránnyal ellenkező előjelű különbséget
    kapunk vagy nem lesz különbség
  \item
    Feltételezzük, hog a null hipotézis igaz!
  \end{itemize}
\item
  A hipotézis tesztelésére alkalmas adatokat gyűjtünk
\item
  Az alternatív hipotézisünket leíró statisztikai modellt illesztjük az
  adatokra

  \begin{itemize}
  \item
    Célunk: az alternatív hipotézis mellett szóló evidencia
    számszerüsítése az adatokban található variancia ellenében
  \item
    A modell illesztése során kiszámoljuk a teszt statisztikát
  \end{itemize}
\item
  Kiszámoljuk a \textbf{teszt statisztikát}

  \begin{itemize}
  \tightlist
  \item
    A teszt statisztikára gondolhatunk úgy, mint a vizsgált hatás
    mértetének mutatója az adatokban található variancia fényében
  \end{itemize}
\item
  Megvizsgáljuk, hogy milyen valószínűséggel kapnánk ilyen vagy ennél
  extrémebb teszt statisztikát, ha a null hipotézis igaz

  \begin{itemize}
  \item
    Ehhez egy valószínűségi eloszlást használunk, ami megmutatja, hogy a
    null hipotézis alatt, milyen valószínűségeket várnánk el az egyes
    teszt statisztika értékekhez
  \item
    Ehhez általában egy elméleti eloszlást használnuk, ami illik a
    vizsgált teszt statisztikához

    \begin{itemize}
    \tightlist
    \item
      Például t tesztnél a teszt statisztika a t érték, és t-eloszlást
      használnuk
    \end{itemize}
  \item
    Az így kiszámolt valószínűség a \textbf{p-érték}
  \end{itemize}
\item
  Értelmezzük az eredmények \textbf{statisztikai szignifikanciáját}

  \begin{itemize}
  \item
    A p-értéket fogjuk használni arra, hogy eldöntsük elég meglepőek-e
    az adataink ahhoz, hogy elvessük a null hipotézist
  \item
    Ehhez egy \textbf{döntési kritériumot} kell használnunk, amit a
    \textbf{szignifikancia küszöbérték}, azaz \textbf{alpha} fog
    számszerűsíteni
  \item
    Fontos megjegyezni, hogy nem tudhatjuk, hogy a döntés, hogy elvetjük
    a null hipozétist helyes vagy helytelen döntés-e egy adott esetben!

    \begin{itemize}
    \tightlist
    \item
      Vagy egyik vagy a másik nem rendelhetünk mellé valószínűséget!
    \end{itemize}
  \item
    Így egyedül azt tudjuk megmondani, hogyha ilyen vagy ennél extrémebb
    adatokat kaptunk hosszútávon, ha végtelenszer megismételjük a
    kísérletet és ugyanezt a döntési kritériumot (alphat) használjuk,
    akkor az esetek hány százalékában fogunk hibásan dönteni
  \end{itemize}
\end{itemize}

\hypertarget{szignifikancia-kuxfcszuxf6b}{%
\subsection{Szignifikancia küszöb}\label{szignifikancia-kuxfcszuxf6b}}

\begin{itemize}
\item
  Nincs objektíven helyes küszöbérték!
\item
  Hagyományosan a szignifikancia küszöb 0.05

  \begin{itemize}
  \item
    Az alpha = 0.05 azt mondja meg, hogyha végtelenszer megismételjük a
    kutatást hosszútávon 5\% az esélye annak, hogy egyes fajú hibát
    követünk el
  \item
    Ez azonban csak tradíció kérdése
  \item
    Pár éve megjelent két cikk is, ami a szignifikancia küszöb
    újragondolására bíztat

    \begin{itemize}
    \item
      Az egyik amellett érvel, hogy hozzunk érveket amellett, hogy
      melyik szignifikancia küszöböt választjuk

      \begin{itemize}
      \tightlist
      \item
        Ez azonban nehéz feladat
      \end{itemize}
    \item
      A másik amellett érvel, hogy használjunk egy konzervatívabb
      küszöbértéket alapértelmezettként, ami legyen 0.005

      \begin{itemize}
      \tightlist
      \item
        Emögött az a feltételezés áll, hogy a 0.05-ös küszöbérték
        mellett a nullhipotézissel szembeni evidencia gyenge
      \end{itemize}
    \end{itemize}
  \end{itemize}
\end{itemize}

\hypertarget{mit-uxe9rtuxfcnk-helyes-duxf6ntuxe9s-alatt}{%
\subsection{Mit értünk helyes döntés
alatt?}\label{mit-uxe9rtuxfcnk-helyes-duxf6ntuxe9s-alatt}}

\begin{itemize}
\item
  Négy lehetséges kimenetel létezik:

  \begin{itemize}
  \item
    Elvetjük a nullhipotézist, amikor az valóban hamis
  \item
    Nem vetjük el a nullhipotézist, amikor valójában igaz
  \item
    Elvetjük a nullhipotézist, pedig az valójában igaz \textbf{(egyes
    fajú hiba)}
  \item
    Nem vetjük el a nullhipotézist, amikor az valójában hamis
    (\textbf{kettes fajú hiba)}
  \end{itemize}
\end{itemize}

\begin{itemize}
\item
  A kettes fajú hiba előfordulásának gyakoriságát a béta küszöbértékkel
  tudjuk kontrollálni

  \begin{itemize}
  \tightlist
  \item
    Általában 0.2 azaz 20\%
  \end{itemize}
\item
  Ha az adataink kellő mértékben valószínűtlenek a null hipotézis
  fényében, akkor elvetjük azt és az alternatív hipotézist megtartjuk
\item
  Ha nem elég valószínűtlenek az adatok, akkor nem tudjuk elvetni a
  nullhipotézist

  \begin{itemize}
  \tightlist
  \item
    Ebben az esetben nem tudhatjuk, hogy a populációban valóban nem
    létezik a hatás vagy csak nem volt elég érzékeny a mérőeszközünk a
    detektálására!
  \end{itemize}
\end{itemize}

\hypertarget{tuxf6bbszuxf6ruxf6s-teszteluxe9s}{%
\subsection{Többszörös
tesztelés}\label{tuxf6bbszuxf6ruxf6s-teszteluxe9s}}

\begin{itemize}
\item
  Ha többször teszteljük ugyanazokat az adatokat a p érték devalválódik
\item
  Nem az egyes tesztekre kell kontrollálni a hibát, hanem a tesztek
  egész családjára

  \begin{itemize}
  \tightlist
  \item
    Ezt nevezzük family wise error rate-nek
  \end{itemize}
\item
  Ezt különböző statisztikai módszerekkel tudjuk kontrollálni

  \begin{itemize}
  \tightlist
  \item
    A legalapvetőbb a Bonferroni korrekció ahol elosztjuk az alphát a
    tesztek számával
  \end{itemize}
\item
  Részben ez magyarázza, hogy miért nem lehet az NHST alatt többszörösen
  tesztelni: azaz megnézni az eredményt újabb adatokat gyűjteni majd
  megint tesztelni
\item
  Hosszutávon a p érték biztosan szignifikáns lesz

  \begin{itemize}
  \tightlist
  \item
    Mert a nagy elemszám miatt a szórás annyira kicsire csökken, hogy
    kis különbséget is szignifikánsnak fogunk találni
  \end{itemize}
\end{itemize}

\hypertarget{hatuxe1snagysuxe1gok}{%
\section{Hatásnagyságok}\label{hatuxe1snagysuxe1gok}}

\begin{itemize}
\item
  A statisztikai szignifikancia nem feltétlenül jelent valós
  szignifikanciát

  \begin{itemize}
  \item
    Példa: egy szívritmus szabályozó gyógyszer hatékonyságának
    vizsgálatánál kaphatunk szignifikáns különbséget, amikor
    összehasonlítjuk azon résztvevők szívritmusát akik kaptak az új
    gyógyszerből azokéval, akik csak placebót kaptak

    \begin{itemize}
    \tightlist
    \item
      ettől függetlenül, ha a gyógyszer klinikailag nem értelmezhető
      módon csökkenti a szívritmus gyakoriságát (mondjuk egy
      gyógynövényből készült tea hatékonyságával ér fel) a gyógyszercég
      nem fog pénzt költeni az új gyógyszer piacra dobására
    \end{itemize}
  \end{itemize}
\item
  A \textbf{hatásméret (effect size)} annak a számszerűsített mutatója,
  mennyire erős egy összefüggés, vagy mennyire nagy a különbség két
  csoport között
\item
  Vannak \textbf{nyers hatásméret mutatók}

  \begin{itemize}
  \item
    Amik ugyanazt a skálát használják, mint az adatok
  \item
    Különböző skálákon mért hatásméretek így nem összehasonlíthatók
  \item
    Azonban a hatás valós szignifikanciájának értelmezése könnyebb
    velük!
  \end{itemize}
\item
  És \textbf{standardizált hatásméret mutatók}

  \begin{itemize}
  \item
    Lehetővé teszi az összehasonlítást!

    \begin{itemize}
    \tightlist
    \item
      Lásd: metaanalízis
    \end{itemize}
  \item
    Két család:

    \begin{itemize}
    \item
      Csoportok közötti különbségekre

      \begin{itemize}
      \tightlist
      \item
        Cohen's d
      \end{itemize}
    \item
      Változók közötti összefüggésre

      \begin{itemize}
      \tightlist
      \item
        Korreláció, odds ratio, risk ratio
      \end{itemize}
    \end{itemize}
  \end{itemize}
\end{itemize}

\hypertarget{mintamuxe9ret-uxe9s-annak-becsluxe9se}{%
\section{Mintaméret és annak
becslése}\label{mintamuxe9ret-uxe9s-annak-becsluxe9se}}

\begin{itemize}
\item
  \textbf{Mintaméret:} az egymástól független megfigyelések száma
\item
  Befolyásolja a statisztikai erőt: annak valószínűsége, hogy
  megtaláljuk a hatást, amennyiben az a valóságban is létezik

  \begin{itemize}
  \tightlist
  \item
    A kettes fajú hiba inverze
  \end{itemize}
\item
  Negatív eredmények esetében fontos a \textbf{statisztikai erő}

  \begin{itemize}
  \tightlist
  \item
    Különben nem tudjuk megmondani hogy valóban nem létezik a hatás vagy
    csak nem volt elég nagy a mintánk, hogy egy ekkora hatást észleljünk
    (nem volt elég érzékeny a mérésünk)
  \end{itemize}
\item
  A szükséges \textbf{mintaméret becslést} az NHST alatt a priori
  mintaméret becsléssel végezzük

  \begin{itemize}
  \item
    Az a priori azt jelenti, hogy csak akkor van értelme a mintaméret
    becslésnek, ha az az adatok gyűjtése és a statisztikai teszt
    elvégzése előtt történt

    \begin{itemize}
    \item
      Ezután vagy megtaláltuk a hatást vagy nem, a kérdés el van döntve
    \item
      Olyan a post hoc mintaméret becslés mintha azután próbálnánk meg
      kiszámolni, hogy mennyi üzemanyag kell egy repülőbe ahhoz, hogy
      átrepüljünk vele az óceán felett, miután már megtörtént a repülés.
      A gép vagy lezuhant vagy nem
    \end{itemize}
  \item
    Ehhez kell az elvárt hatásméret, a kívánt statisztikai erő, a
    szignifikancia küszöb, a kutatási elrendezés
  \item
    Minél kisebb az elvárt hatás annál nagyobb mintára lesz szükségünk,
    hogy észleljük
  \item
    Az elvárt hatásméret megbecslésére számos módszer van

    \begin{itemize}
    \item
      Használhatunk pilot kutatást
    \item
      Támaszkodhatunk a szakirodalomban előzetesen talált hatásméretekre

      \begin{itemize}
      \tightlist
      \item
        Azonban itt lehetséges, hogy a publikált hatásméretek
        torzítanak!
      \end{itemize}
    \item
      Elméleti alapon is meghatározhatjuk a számunkra releváns legkisebb
      hatást

      \begin{itemize}
      \tightlist
      \item
        \textbf{smallest effect size of interest (SESOI)}
      \item
        melyik az a legkisebb hatás, ami számunkra még érdekes lehet
      \end{itemize}
    \end{itemize}
  \end{itemize}
\end{itemize}

\bookmarksetup{startatroot}

\hypertarget{kuxe9t-folytonos-vuxe1ltozuxf3-kuxf6zti-uxf6sszefuxfcgguxe9s}{%
\chapter{Két folytonos változó közti
összefüggés}\label{kuxe9t-folytonos-vuxe1ltozuxf3-kuxf6zti-uxf6sszefuxfcgguxe9s}}

\begin{itemize}
\item
  Statisztikai próbája a \textbf{korreláció}
\item
  Teszt értéke a \textbf{korrelációs együttható} \textbf{(r)}

  \begin{itemize}
  \item
    skálafüggetlen~
  \item
    -1 és 1 közötti értéket vehet fel a korreláció irányának
    függvényében
  \end{itemize}
\item
  Lineáris összefüggést feltételez a két változó között

  \begin{itemize}
  \item
    Pozitív: egyik változó nő a másik is nő
  \item
    Negatív: egyik változó nő a másik csökken
  \end{itemize}
\item
  Összefüggés erősségeként lehet értelmezni az r nagyságát

  \begin{itemize}
  \tightlist
  \item
    Minél nagyobb annál erősebb az összefüggés
  \end{itemize}
\item
  3 típusa van

  \begin{itemize}
  \item
    \textbf{Pearson korreláció:}

    \begin{itemize}
    \item
      Legalább intervallum típusú változók
    \item
      Változók eloszlása normál eloszlást követ
    \end{itemize}
  \item
    \textbf{Spearman rang korreláció:}

    \begin{itemize}
    \tightlist
    \item
      Legalább ordinális típusú változók
    \end{itemize}
  \item
    \textbf{Kendall tau:}

    \begin{itemize}
    \item
      Legalább ordinális
    \item
      Alacsonyabb mintaelemszámnál is jó
    \item
      Jól kezeli ha sok azonos rangú elem van
    \end{itemize}
  \end{itemize}
\item
  Korreláció null hipotézis tesztje

  \begin{itemize}
  \item
    Kritikus értéknél az r-t t-értékké transzformáljuk
  \item
    Szabadságfok: n-2
  \item
    Excelben nem annyira egyszerű kiszámítani a korrelációs
    együtthatóhoz tartozó p-értéket, ezért ezen az órán ettől
    eltekintünk és a korrelációs együttható, mint két folytonos változó
    közötti kapcsolat erősségét és irányát kifejező mutató értelmezésére
    koncentrálunk
  \end{itemize}
\item
  \textbf{Korrelációs mátrix}

  \begin{itemize}
  \tightlist
  \item
    Több változó között egyszerre mutatja meg a korrelációs
    együtthatókat táblázat formájában
  \end{itemize}
\item
  Korrelációs együttható négyzete a \textbf{determinisztikus együttható
  (R\textsuperscript{2})}

  \begin{itemize}
  \item
    Egyik változó mekkora részt magyaráz a másik változóból
  \item
    A variance mekkora részét magyarázza meg
  \item
    Regressziónál fontos mutató
  \end{itemize}
\end{itemize}

\bookmarksetup{startatroot}

\hypertarget{lineuxe1ris-regressziuxf3}{%
\chapter{Lineáris regresszió}\label{lineuxe1ris-regressziuxf3}}

\hypertarget{egyszerux171-lineuxe1ris-regressziuxf3}{%
\section{Egyszerű lineáris
regresszió}\label{egyszerux171-lineuxe1ris-regressziuxf3}}

\begin{itemize}
\item
  Legsokrétűbb statisztikai próba
\item
  A legtöbb más statisztikai próbát regresszióként is lehet értelmezni
\item
  Célja: a kimeneti változó értékeinek predikciója egy vagy több
  prediktor változó által
\item
  Adat = model + error

  \begin{itemize}
  \tightlist
  \item
    Lineáris regresszió esetében a modellünk egy egyenes vonal
  \end{itemize}
\item
  Azt a lineáris modellt akarjuk megtalálni, ami legjobban illeszkedik
  az adatokra, ahol legkisebb a hiba
\end{itemize}

\hypertarget{lineuxe1ris-modell-paramuxe9terei}{%
\subsection{Lineáris modell
paraméterei}\label{lineuxe1ris-modell-paramuxe9terei}}

\begin{itemize}
\item
  Másnéven regressziós \textbf{együtthatók (regression coefficients)}
\item
  \textbf{Meredekség (slope):} egy egységnyi változás a prediktor
  változóban (x tengely) mekkora változást okoz a kimeneti változóban (y
  tengely)

  \begin{itemize}
  \item
    Ha pozitív érték, akkor pozitív kapcsolat van a két változó között

    \begin{itemize}
    \tightlist
    \item
      Ha a prediktor változónk értéke nő a kimeneti változó értéke is nő
    \end{itemize}
  \item
    Ha negatív érték, akkor negatív kapcsolat

    \begin{itemize}
    \tightlist
    \item
      Ha a prediktor változónk értéke nő, a kimeneti változónk értéke
      csökken
    \end{itemize}
  \item
    Jele: b1
  \end{itemize}
\item
  \textbf{Intercept:} ha a prediktor változó értéke 0, mekkora a
  kimeneti változó értéke, azaz milyen y értéknél metszi a vonal az y
  tengelyt

  \begin{itemize}
  \tightlist
  \item
    Jele: b0
  \end{itemize}
\end{itemize}

\hypertarget{lineuxe1ris-modell-illeszkeduxe9suxe9nek-a-vizsguxe1lata}{%
\subsection{Lineáris modell illeszkedésének a
vizsgálata}\label{lineuxe1ris-modell-illeszkeduxe9suxe9nek-a-vizsguxe1lata}}

\begin{itemize}
\item
  Hogyan találjuk meg azt a lineáris modellt, ami legjobban illeszkedik
  az adatokra?

  \begin{itemize}
  \item
    A \textbf{legkisebb négyzetek módszere (method of least squares)}

    \begin{itemize}
    \item
      Meghatározásához ugyanazt a módszert használhatjuk, mint amikor az
      átlagot használtuk, mint modellt
    \item
      Megnézzük a modell által prediktált értékek és a valós értékek
      közötti különbséget: a reziduálisokat (residuals)
    \item
      Itt is négyzetre emeljük a reziduálisok majd összeadjuk őket (sum
      of squared differences, SS)
    \item
      Ezután minden lehetséges vonalra (lineáris regressziós modellre)
      kiszámíthatnánk őket és ahol az SS a legkisebb az a modell
      illeszkedik a legjobban az adatokra
    \item
      Azonban még a legjobban illeszkedő modell is magyarázhatja rosszul
      az adatokat!
    \end{itemize}
  \item
    Ahhoz, hogy ezt megvizsgálájuk a legegyszerűbb modell
    illeszkedéséhez hasonlítjuk a regressziós modellunk illeszkedését:
    az átlaghoz

    \begin{itemize}
    \item
      Az átlag azonban minden adatpontra ugyanazt az értéket fogja
      prediktálni

      \begin{itemize}
      \item
        Például: mennyire számít a marketingre szánt összeg egy film
        összbevételénél?

        \begin{itemize}
        \tightlist
        \item
          Ha 1 dollárt költünk a marketingre akkor is ugyanazt a
          bevétlet prediktálja, mintha 200000\$ költüttünk volna rá
        \end{itemize}
      \end{itemize}
    \item
      Az átlag modellnél a hibát összesítve megkapjuk a \textbf{teljes
      négyzetösszeget (total sum of squares, SS\textsubscript{T})}

      \begin{itemize}
      \tightlist
      \item
        Ez a teljes, mert ez a legegyszerűbb modell
      \end{itemize}
    \item
      Kiszámítjuk a \textbf{négyzetre emelt reziduálisok összegét} a
      regressziós modellnél is \textbf{(residual sum of squares,
      SS\textsubscript{R})}
    \item
      Ahhoz, hogy megtudjuk a regressziós modellünk mennyivel jobban
      magyarázza az adatokat, mint az átlag modellunk a kettőt kivonjuk
      egymásból: SS\textsubscript{T} - SS\textsubscript{R} =
      \textbf{SS\textsubscript{M} (model sum of squares)}

      \begin{itemize}
      \tightlist
      \item
        Ha az SS\textsubscript{M} nagy, a regressziós modell sokkal
        jobban magyarázza az adatokat, mint az átlag
      \end{itemize}
    \item
      Megnézhetjük, hogy arányosan mennyivel javul a modellünk az
      átlaghoz képest, ha egy prediktor változót is belerakunk

      \begin{itemize}
      \item
        A kettőt elosztva egymással megkapjuk a \textbf{determinisztikus
        együtthatót R\textsuperscript{2}}

        \begin{itemize}
        \tightlist
        \item
          SS\textsubscript{M} / SS\textsubscript{T}
        \end{itemize}
      \item
        Megmutatja a regressziós modellünk által megmagyarázott
        variancia arányát a kimeneti változónkban, a teljes varianciához
        képest
      \item
        Ha megszorozzuk 100-al százalékot kapunk
      \item
        A kimeneti változóban lévő varianciának hány százalékát
        magyarázza meg a modell
      \item
        Ha ennek a négyzetgyökét vesszük, akkor megkapjuk a Pearson
        korrelációs együtthatót!
      \end{itemize}
    \item
      Az F-teszttel is megvizsgálhatjuk a modellünk illeszkedését

      \begin{itemize}
      \item
        \textbf{F teszt statisztika}: szisztematikus variancia / nem
        szisztematikus variancia

        \begin{itemize}
        \tightlist
        \item
          A modell okozta javulás (SS\textsubscript{M}) / a modell és a
          megfigyel adatok között lévő különbség (SS\textsubscript{R})
        \end{itemize}
      \item
        Másszóval az F teszt statisztika megmondja, hogy a modell
        mennyire javítja a becslésünk pontosságát a modellben található
        pontatlansághoz képest
      \item
        Itt nem a négyzetes különbségek összegével, hanem átlagával
        dolgozunk

        \begin{itemize}
        \tightlist
        \item
          Így nem függ a megfigyelések számától
        \end{itemize}
      \item
        Mean squares for the model (MSS)

        \begin{itemize}
        \tightlist
        \item
          Szabadságfok: prediktor változók száma
        \end{itemize}
      \item
        Residuals mean square (MS\textsubscript{R})

        \begin{itemize}
        \tightlist
        \item
          Szabadságfok: megfigyelések száma - béta együtthatók száma
          (meredség + intercept = 2)
        \end{itemize}
      \item
        Jó modellnél 1-nél nagyobb az F arány
      \item
        p-értéket vagy konfidencia intervallumot is ki tudunk hozzá
        számolni
      \end{itemize}
    \end{itemize}
  \end{itemize}
\end{itemize}

\hypertarget{prediktor-vuxe1ltozuxf3k-szignifikanciuxe1juxe1nak-vizsguxe1lata}{%
\subsection{Prediktor változók szignifikanciájának
vizsgálata}\label{prediktor-vuxe1ltozuxf3k-szignifikanciuxe1juxe1nak-vizsguxe1lata}}

\begin{itemize}
\item
  Nemcsak a teljes modell teljesítményét kell megvizsgálnunk, hanem az
  egyes paramétereknek a szignifikanciáját is
\item
  A béta megmutatja, hogy a prediktorban való egy egységnyi változás
  mekkora változást okoz a kimeneti változóban

  \begin{itemize}
  \item
    Ha a modell rossz, azt várjuk el, hogy ez nulla legyen

    \begin{itemize}
    \tightlist
    \item
      Pont, mint az átlagnál!
    \end{itemize}
  \end{itemize}
\item
  A nullhipotézis a paraméterek esetén az lesz, h a paraméter nem
  különbözik a nullától
\item
  A kritikus érték pedig a paraméter tényleges értéke
\item
  Ezekre gyakorlatilag egy egymintás t-próbát fogunk végezni

  \begin{itemize}
  \tightlist
  \item
    t = béta / SE
  \end{itemize}
\item
  Az intercept esetén, hogy különbözik-e a nullától az érték.
\item
  A slope esetén, hogy a dőlésszöge különbözik-e a nullától.
\end{itemize}

\hypertarget{lineuxe1ris-regressziuxf3-elux151feltuxe9telei}{%
\section{Lineáris regresszió
előfeltételei}\label{lineuxe1ris-regressziuxf3-elux151feltuxe9telei}}

\begin{itemize}
\item
  A kimeneti változó folytonos azaz legalább intervallum mérési szintű
\item
  Prediktor típusok: folytonos vagy kategorikus is lehet
\item
  Nem zéró variancia: a kimeneti változó és prediktor értékeiben van
  variabilitás
\item
  A megfigyelések egymástól függetlenek
\item
  A reziduálisok eloszlása normális (a prediktor eloszlásának nem kell
  normálisnak lennie!)

  \begin{itemize}
  \item
    Vizuálisan:~

    \begin{itemize}
    \item
      \textbf{QQ plot}

      \begin{itemize}
      \item
        A pontok maradjanak az átló közelében

        \begin{itemize}
        \item
          Az esetek 5\%-a lehet 2 szóráson kívül
        \item
          Az esetek 1\%-a lehet 2.5 szóráson kívül
        \item
          Az esetek 0.1\%-a lehet 3 szóráson kívül
        \end{itemize}
      \end{itemize}
    \item
      \textbf{Residual vs fitted values}

      \begin{itemize}
      \item
        Residuális értékek vannak az y tengelyen
      \item
        Modell értékek az x tengelyen
      \item
        Ha a vonal görbül, nem lineáris kapcsolat
      \end{itemize}
    \end{itemize}
  \end{itemize}
\item
  Az értékek

  \begin{itemize}
  \item
    68\%-a egy szóráson belül van
  \item
    95\%-a két szóráson belül van
  \item
    99.7\%-a három szóráson belül van
  \end{itemize}
\item
  Ehhez kapcsolódik, hogy a modellben nincsen sok jelentős kiugró érték
  (outlier), ami torzítja a modellünket
\item
  Azt várjuk, hogy a lineáris regresszió minden mérési szinten
  ugyanannyira jó predikciót tudjon adni. Azaz, ugyanolyan hatékony
  legyen akkor, ha a buszmegállóban 3 ember van, mint akkor, ha 20
\item
  Ezt a reziduálisok elemzésével tudjuk ellenőrizni
\item
  Ekkor azt mondjuk, hogy a modellünk homoszkedasztikus, azaz a
  reziduálisok mértéke független a prediktor értékétől.
\item
  Ellentéte a heteroszkedaszticitás, ami azt jelenti, hogy pl. a kisebb
  prediktált értékekhez tartozó reziduálisok kisebbek, mint a nagyobb
  prediktált értékekhez tartozó reziduálisok
\item
  Vizsgálata vizuálisan zajlik

  \begin{itemize}
  \tightlist
  \item
    Tölcsér alak azt jelenti hogy sérül a heteroszkedaszticitás feltétle
  \end{itemize}
\item
  Kiugró értékek szűrése

  \begin{itemize}
  \item
    Vizuálisan

    \begin{itemize}
    \item
      Távol esnek a többi értéktől
    \item
      Magukhoz húzzák a regressziós egyenest
    \end{itemize}
  \item
    Statisztikai módszerekkel

    \begin{itemize}
    \item
      Cook's distance
    \item
      Ha 1-nél nagyobb erős torzító hatása van az adatpontnak
    \end{itemize}
  \item
    Mit tegyünk ha vannak outlierek?

    \begin{itemize}
    \item
      Csak akkor zárjuk ki ha adathibából származnak
    \item
      Különben overfitting veszélye fennáll
    \end{itemize}
  \item
    Nagy elemszánál nem olyan nagy a hatásuk!
  \end{itemize}

  \hypertarget{tuxf6bbszuxf6ruxf6s-lineuxe1ris-regressziuxf3}{%
  \section{Többszörös lineáris
  regresszió}\label{tuxf6bbszuxf6ruxf6s-lineuxe1ris-regressziuxf3}}

  \begin{itemize}
  \item
    További tényezőkről is gondolhatjuk, hogy javítani fognak a
    modellünkön

    \begin{itemize}
    \item
      Egy bizonyos pont után, ha ezeket a prediktor változókat
      hozzáadjuk a modellhez, nem fog a hiba szignifikánsan csökkenni
    \item
      Ez az overfitting
    \item
      Eredménye: a modell nem lesz generalizálható
    \end{itemize}
  \end{itemize}
\end{itemize}

\bookmarksetup{startatroot}

\hypertarget{kuxe9t-csoport-uxe1tlaguxe1nak-uxf6sszehasonluxedtuxe1sa}{%
\chapter{Két csoport átlagának
összehasonlítása}\label{kuxe9t-csoport-uxe1tlaguxe1nak-uxf6sszehasonluxedtuxe1sa}}

\begin{itemize}
\item
  Két csoportot szeretnénk összehasonlítani ahelyett, hogy változók
  közötti összefüggés erősségét vizsgálnánl
\item
  Általában kísérleti elrendezésnél használjuk

  \begin{itemize}
  \item
    Itt a kísérleti elrendezés lehetővé teszi, hogy ok-okozati
    összefüggést állítsunk fel
  \item
    Kétféle módon gyűjthetünk adatokat

    \begin{itemize}
    \item
      Különböző embereket véletlenszerűen kísérleti csoportokba
      sorolhatunk

      \begin{itemize}
      \item
        Minden ember csak egy csoportba kerül
      \item
        \textbf{Független méréses elrendezés}
      \end{itemize}
    \item
      Minden résztvevő végigmegy az összes kísérleti manipuláción

      \begin{itemize}
      \item
        \textbf{Többszörös méréses elrendezés}
      \item
        Előnye lehet, hogy a személyek közötti esetleges torzító
        tényezők hatását kontrollálni tudjuk

        \begin{itemize}
        \tightlist
        \item
          Ezzel szenzitívebb lesz a mérésünk
        \end{itemize}
      \end{itemize}
    \item
      A kettőt egyszerre is alkalmazhatjuk

      \begin{itemize}
      \tightlist
      \item
        \textbf{Kevert elrendezés}
      \end{itemize}
    \item
      A kísérleti elrendezéstől függ, hogy milyen statisztikai elemzési
      módszert kell használnunk!

      \begin{itemize}
      \tightlist
      \item
        Hiszen ismételt mérések esetén a csoportok nem lesznek
        függetlenek egymástól
      \end{itemize}
    \end{itemize}
  \end{itemize}
\end{itemize}

\hypertarget{a-t-pruxf3ba-tuxedpusai}{%
\section{A t-próba típusai}\label{a-t-pruxf3ba-tuxedpusai}}

\begin{itemize}
\tightlist
\item
\end{itemize}

\hypertarget{a-t-pruxf3ba-logikuxe1ja}{%
\section{A t-próba logikája}\label{a-t-pruxf3ba-logikuxe1ja}}

\begin{itemize}
\item
  Két mintát veszünk, ahol a mintáink lehetőleg csak az átlalunk
  vizsgált független változó mentén különböznek szisztematikusan

  \begin{itemize}
  \item
    Pl: A) csoport tagjai részt vettek egy képzésen b) csoport tagjai
    nem vettek rész a képzésen (kontroll csoport)

    \begin{itemize}
    \item
      Célunk, hogy a csoport tagjai lehetőleg minden másban hasonlóak
      legyenek

      \begin{itemize}
      \tightlist
      \item
        Ne legyen olyan esetleges torzító faktor (például A) csoportba
        csak magas IQ-val rendelkező személyek vannak), ami torzíthatja
        a vizsgált hatást (a képzés hatására jobb pontot érnek-e el a
        személyek egy teszten)
      \end{itemize}
    \end{itemize}
  \end{itemize}
\item
  A null hipotézis alatt azt várjuk el, hogy a két minta átlagai között
  nem lesz különbség, a csoportok átlagainak különbsége a függő
  változóban a nulla körül helyezkedik el

  \begin{itemize}
  \tightlist
  \item
    Ebben az esetben a két minta feltételezhetően ugyanabból a
    populációból származik
  \end{itemize}
\item
  A megfigyelt minták átlagainak különbségét ahhoz a hipotetikus
  különbség értékhez hasonlítjuk, amit akkor várnánk el, ha nincs hatás
  (tehát a null hipotézis igaz)
\item
  A standard hibát használjuk arra, hogy megvizsgáljuk a minta átlagok
  közti variabilitást

  \begin{itemize}
  \item
    Ha a standard hiba kicsi akkor várhatóan a minta átlagok hasonlóak
    lesznek
  \item
    Ha a standard hiba nagy akkor várhatóan nagy különbségek is
    előfordulhatnak a minta átlagok között
  \item
    Még akkor is, ha ugyanabból a populációból származnak
  \end{itemize}
\item
  Ha a megfigyelt minta átlagok közti különbség nagyobb, mint amit
  elvárnánk a standard hiba alapján, ha a null hipotézis igaz, akkor
  feltételezhetjük:

  \begin{itemize}
  \item
    A valóságban nincs hatás, a két minta átlag ugyanabból a
    populációból származik, csak atipikus egyedei annak

    \begin{itemize}
    \tightlist
    \item
      Másszóval: távol helyezkednek el a populáció átlagtól
    \end{itemize}
  \item
    A valóságban van hatás, a null hipotézist elvetjük, a két minta
    valójában két különböző populációból származik, amelyeknek tipikus
    tagjai

    \begin{itemize}
    \tightlist
    \item
      Másszóval a minta átlagok közel vannak a populációk átlagához,
      amiből származnak
    \end{itemize}
  \end{itemize}
\item
  Minél nagyobb a hatás és minél kisebb a standard hiba, annál
  biztosabbak vagyunk benne, hogy a hatás valóban létezik
\end{itemize}

\hypertarget{t-pruxf3ba-mint-stasztikai-modell}{%
\section{t-próba, mint stasztikai
modell}\label{t-pruxf3ba-mint-stasztikai-modell}}

\begin{itemize}
\item
  Egy teszt statisztikára gondolhatunk úgy is, mint a variancia, amit
  megmagyaráz a modellunk elosztva a varianciával, amit nem magyaráz meg
  a modellünk

  \begin{itemize}
  \tightlist
  \item
    hatás / hiba
  \end{itemize}
\item
  t-próba esetében a statisztikai modell a két csoport átlagainak a
  különbsége
\item
  A hiba mutatónk a standard hiba

  \begin{itemize}
  \item
    Megmutatja mennyire fluktuálnak a minta átlagok a mintevételezési
    hibának következtében
  \item
    Ezt is az átlagok különbségére számoljuk ki
  \end{itemize}
\end{itemize}

\hypertarget{t-pruxf3ba-elux151feltuxe9telei}{%
\section{t-próba előfeltételei}\label{t-pruxf3ba-elux151feltuxe9telei}}

\begin{itemize}
\item
  A változók legalább intervallum skálájúak

  \begin{itemize}
  \tightlist
  \item
    Ez az előfeltétele, hogy átlagot tudjunk számolni
  \end{itemize}
\item
  A mintavételei eloszlás (sampling distribution) normáls eloszlást
  követ

  \begin{itemize}
  \item
    A mintavételi eloszlást nem tudjuk közvetlenül megvizsgálni ezért
    azt nézzük meg, hogy a változók eloszlása normál eloszlást követ-e

    \begin{itemize}
    \item
      A centrális határeloszlás elve miatt feltételezzük

      \begin{itemize}
      \item
        ha a változónk eloszlása normális a mintavételi eloszlás is
        normális lesz
      \item
        Illetve sok megfigyelés esetén a mintavételezési eloszlás úgyis
        a normálishoz közelít
      \end{itemize}
    \end{itemize}
  \item
    Páros t-próba esetén a két csoport értékeinek a különbsége kell
    normál eloszlást kövessen!
  \end{itemize}
\item
  Független mintás t-próba

  \begin{itemize}
  \item
    Az előfeltételeket tesztelni kell mielőtt a parametrikus Student
    t-próbát alkalmazhatnánk
  \item
    Az előző feltételeken felül:

    \begin{itemize}
    \item
      A különböző csoportokból származó értékek függetlenek egymástól
    \item
      Szórás homogenitás: a két csoport szórása hasonló
    \end{itemize}
  \end{itemize}
\end{itemize}

\hypertarget{t-pruxf3ba-eredmuxe9nyeinek-uxe1bruxe1zoluxe1sa}{%
\section{t-próba eredményeinek
ábrázolása}\label{t-pruxf3ba-eredmuxe9nyeinek-uxe1bruxe1zoluxe1sa}}

\begin{itemize}
\item
  Hegedű ábrával (violin plot)
\item
  Barchart-tal
\item
  Ne feledkezzünk meg az error bar-ról!

  \begin{itemize}
  \tightlist
  \item
    Ez általában a standard hiba (SE)
  \end{itemize}
\end{itemize}

\bookmarksetup{startatroot}

\hypertarget{hivatkozuxe1sok}{%
\chapter*{Hivatkozások}\label{hivatkozuxe1sok}}
\addcontentsline{toc}{chapter}{Hivatkozások}

\markboth{Hivatkozások}{Hivatkozások}

\hypertarget{refs}{}
\begin{CSLReferences}{0}{0}
\end{CSLReferences}



\end{document}
